\documentclass{uetgraduation}

\usepackage[hidelinks]{hyperref}
\setlength{\parindent}{2em}
\usepackage{array}      % mở rộng tùy chọn cột
\usepackage{tabularx}   % cột tự co giãn X
\usepackage{adjustbox}  % thu-phóng bảng theo bề rộng
\usepackage{algorithm}
\usepackage{algpseudocode}
\usepackage{amsfonts}   % hoặc
\usepackage{amssymb}

% Nếu cần định nghĩa thêm các kiểu đính kèm (ví dụ cho code, sơ đồ), bạn có thể đặt ở đây
% \makeattachmenttype{...}{...}{...}{...}{...}{...}{...}{...}

\begin{document}

% ===========================
% THÔNG TIN TRANG BÌA
% ===========================
\studentname{Trần Hồng Quân}
\title{NNHẬN DIỆN HOẠT ĐỘNG CỦA CÁC THIẾT BỊ ĐIỆN PHỨC TẠP TRONG MẠNG ĐIỆN SỬ DỤNG THUẬT TOÁN PHÁT HIỆN SỰ KIỆN ĐƯỢC TỐI ƯU}
\documenttype{Đồ án tốt nghiệp chương trình đào tạo chuẩn}
\major{Kỹ thuật máy tính}
\year{2025}
\supervisor{TS. Nguyễn Ngọc An}
\title{NHẬN DIỆN HOẠT ĐỘNG CỦA CÁC THIẾT BỊ ĐIỆN PHỨC TẠP TRONG MẠNG ĐIỆN SỬ DỤNG THUẬT TOÁN PHÁT HIỆN SỰ KIỆN ĐƯỢC TỐI ƯU}
\major{Kỹ thuật máy tính}
\supervisor{TS. Nguyễn Ngọc An}
\makecovers

% ===========================
% TÓM TẮT, CAM ĐOAN, CẢM ƠN
% ===========================
\begin{preamble}{Tóm tắt}
\textbf{Tóm tắt:} 
Trong bối cảnh nhu cầu quản lý năng lượng thông minh ngày càng gia tăng, việc giám sát và nhận dạng mức tiêu thụ điện của từng thiết bị trở thành yếu tố quan trọng để tối ưu hóa sử dụng điện và giảm lãng phí năng lượng. Một trong những hướng tiếp cận phổ biến để thực hiện điều này là NILM (Non-Intrusive Load Monitoring) — phương pháp giám sát tải điện không xâm lấn, cho phép xác định mức tiêu thụ của từng thiết bị chỉ từ dữ liệu tổng hợp đo tại công tơ điện mà không cần gắn cảm biến lên từng thiết bị.

Mặc dù NILM đã được nghiên cứu trong nhiều năm, các hệ thống hiện tại vẫn còn nhiều hạn chế. Phần lớn nghiên cứu chỉ tập trung vào một phần của bài toán — như phát hiện sự kiện bật/tắt hoặc phân loại thiết bị — mà chưa tích hợp đầy đủ hai nhiệm vụ này trong một quy trình thống nhất. Bên cạnh đó, nhiều mô hình còn giả định môi trường đơn giản hoặc gần như lý tưởng để dễ dàng triển khai, nhưng những giả định này không phù hợp với thực tế, nơi nhiều thiết bị có thể hoạt động đồng thời, gây chồng lấn tín hiệu và dẫn đến bỏ sót hoặc nhận dạng sai thiết bị.

Trước những hạn chế này, đề tài tập trung xây dựng và kiểm chứng một thuật toán NILM hướng sự kiện có khả năng hoạt động ổn định trong môi trường phức tạp, phát hiện chính xác các sự kiện bật/tắt của nhiều thiết bị hoạt động đồng thời và nhận diện chính xác thiết bị. Dữ liệu điện áp và dòng điện được thu thập từ nhiều thiết bị khi hoạt động trong môi trường thực tế và triển khai thuật toán đề xuất trên dữ liệu đo được. Thuật toán thực hiện các bước chính: Phát hiện sự kiện bật/tắt, tiền xử lý và trích xuất đặc trưng (bao gồm công suất tức thời và đặc trưng ảnh dạng sóng I–V), và nhận dạng thiết bị bằng mô hình học máy MLP kết hợp giữa các đặc trưng công suất và hình ảnh tín hiệu.  

Đề tài cũng xây dựng một bộ dữ liệu thực nghiệm mới, ghi nhận các sự kiện bật/tắt của nhiều thiết bị trong điều kiện thực tế tại Việt Nam, làm cơ sở kiểm thử trực tiếp cho hệ thống NILM hướng sự kiện. Các thuật toán phát hiện sự kiện như WAMMA hay thuật toán Hybrid của Mengqi Lu – Zuyi Li cũng được triển khai để làm cơ sở so sánh với thuật toán mà đề tài đề xuất, qua đó đánh giá và chứng minh hiệu quả của phương pháp mới.

Kết quả thực nghiệm cho thấy thuật toán đề xuất có khả năng phát hiện sự kiện bật/tắt và nhận dạng thiết bị với độ tin cậy cao trong môi trường nhiều thiết bị hoạt động đồng thời. Đề tài đóng góp hai kết quả chính: (1) Xây dựng thành công thuật toán NILM hướng sự kiện, (2) Phát triển bộ dữ liệu thực nghiệm phục vụ kiểm thử và nghiên cứu NILM, đồng thời tạo nền tảng cho các ứng dụng thực tế trong nhà thông minh, quản lý năng lượng và giám sát tiêu thụ điện trong công nghiệp.

\textbf{Từ khóa:} NILM, NILM hướng sự kiện, sự kiện bật/tắt

\end{preamble}

\begin{preamble}{Lời cam đoan}
    
Tôi xin cam kết rằng đồ án tốt nghiệp với đề tài “Nhận diện hoạt động của các thiết bị điện phức tạp sử dụng thuật toán phát hiện sự kiện được tối ưu” là kết quả nghiên cứu do chính tôi thực hiện. Toàn bộ nội dung và kết
quả trình bày trong đồ án là trung thực, phản ánh đúng quá trình thực nghiệm và phân tích của
bản thân tôi, không sao chép hoặc sử dụng từ bất kỳ công trình nào đã được công bố hoặc bảo
vệ tại các cơ sở giáo dục khác.
Tôi hoàn toàn chịu trách nhiệm về tính trung thực và nguyên bản của đồ án này trước Hội đồng
chấm đồ án và các quy định liên quan của nhà trường.

\vspace{1.5cm}
\begin{flushright}
    Hà Nội, ngày \hspace{0.5cm} tháng \hspace{0.5cm} năm 2025\\
\end{flushright}
\begin{tabular}{p{0.5\linewidth} p{0.5\linewidth}}
    \hspace{1cm}\textbf{} & \hspace{2cm}\textbf{Sinh viên thực hiện} \\
    \\[2cm] % khoảng trống để ký
    \hspace{1cm}\textbf{} & \hspace{2.1cm}\textbf{Trần Hồng Quân} \\
\end{tabular}

\end{preamble}

\begin{preamble}{Lời cảm ơn}
Để hoàn thành được đồ án này, trước hết tôi xin bày tỏ lòng biết ơn đến thầy TS. Nguyễn Ngọc An, người đã tận tình hướng dẫn, đồng hành và dành nhiều thời gian quý báu để định hướng, hỗ trợ tôi trong suốt quá trình thực hiện đồ án.

Tôi cũng xin gửi lời cảm ơn chân thành đến Ban Giám hiệu Nhà trường cùng các thầy, cô giáo ngành Kỹ thuật máy tính đã không ngừng tạo điều kiện học tập thuận lợi và truyền đạt những kiến thức nền tảng quý giá trong suốt quá trình học tập tại trường.

Dù đã cố gắng hoàn thiện đồ án với tinh thần nghiêm túc và nỗ lực cao nhất, nhưng chắc chắn không tránh khỏi những thiếu sót. Tôi mong nhận được những góp ý chân thành từ các thầy cô
để đồ án được hoàn thiện hơn.

Cuối cùng, tôi xin gửi lời cảm ơn đến nhóm nghiên cứu myLab của thầy Nguyễn Ngọc An đã tiếp thêm động lực để tôi có thể kiên trì theo đuổi và hoàn thành đồ án này.
\end{preamble}

% ===========================
% MỤC LỤC, DANH SÁCH HÌNH, BẢNG
% ===========================
\begin{contentlisting}
\tableofcontents
\listoffigures
\listoftables
\end{contentlisting}

% ===========================
% CÁC CHƯƠNG NỘI DUNG
% ===========================
\chapter{Giới thiệu}

\section{Giới thiệu NILM}

Trong lĩnh vực quản lý năng lượng, giám sát tải điện (load monitoring) là quá trình theo dõi lượng điện tiêu thụ của các thiết bị trong hệ thống. Mục đích của việc giám sát này nhằm tối ưu hóa mức tiêu thụ và giúp người dùng sử dụng điện hiệu quả hơn. Tuy nhiên, các phương pháp giám sát truyền thống thường yêu cầu gắn cảm biến trực tiếp lên từng thiết bị để đo công suất, khiến chi phí triển khai cao và khó áp dụng trong thực tế.

NILM (Non-Intrusive Load Monitoring), hay còn gọi là giám sát tải điện không xâm lấn, là một giải pháp tiên tiến được phát triển để khắc phục những hạn chế đó. Thay vì phải lắp cảm biến cho từng thiết bị, NILM cho phép nhận dạng và ước lượng mức tiêu thụ của từng thiết bị chỉ từ dữ liệu tổng hợp về dòng điện và điện áp tại một điểm đo duy nhất, thường đặt tại công tơ chính của hộ gia đình hoặc tòa nhà \cite{ref1}.

Khác với phương pháp ILM (Intrusive Load Monitoring), trong đó mỗi thiết bị cần một cảm biến đo riêng, NILM chỉ yêu cầu một bộ đo duy nhất. Nhờ vậy, hệ thống NILM mang lại nhiều lợi ích như giảm đáng kể chi phí triển khai, hạn chế tác động lên thiết bị hiện hữu, dễ mở rộng, linh hoạt và phù hợp cho nhiều môi trường từ hộ gia đình, văn phòng đến nhà máy công nghiệp.

NILM có nhiều ứng dụng thực tiễn: Trong nhà thông minh, nó hỗ trợ giám sát mức tiêu thụ của từng thiết bị và cảnh báo khi có dấu hiệu bất thường; trong quản lý năng lượng, nó giúp người dùng tối ưu hóa sử dụng điện và giảm phát thải; trong công nghiệp, NILM cho phép theo dõi hiệu suất thiết bị mà không cần gắn cảm biến rời từng máy.

Với khả năng theo dõi mà không cần can thiệp sâu vào hệ thống điện, NILM được xem là một hướng tiếp cận đầy hứa hẹn trong quản lý năng lượng hiện đại, đồng thời mở ra nhiều cơ hội nghiên cứu trong xử lý tín hiệu, học máy và IoT.

\section{Phân loại thuật toán NILM}

Hiện nay, các thuật toán NILM (Non-Intrusive Load Monitoring) được phát triển theo hai hướng tiếp cận chính là hướng liên tục (Continuous-based NILM) và hướng theo sự kiện (Event-based NILM). Mỗi hướng tiếp cận có triết lý xử lý tín hiệu và ưu, nhược điểm riêng, tùy thuộc vào mục tiêu và điều kiện triển khai của hệ thống

Mỗi hướng có triết lý khác nhau về cách xử lý tín hiệu:

\begin{itemize}
    \item Continuous: “Theo dõi suốt quá trình tiêu thụ”
    \item Event: “Chỉ chú ý khi có thay đổi rõ ràng”
\end{itemize}

\subsection{Hướng tiếp cận liên tục (Continuous-based)}

\textbf{Nguyên lý hoạt động:}

Thuật toán theo hướng này sẽ phân tích toàn bộ chuỗi tín hiệu tổng (dòng điện, điện áp, công suất tức thời, v.v.) trong suốt thời gian hoạt động. Dữ liệu được thu thập liên tục và đưa vào mô hình xử lý để dự đoán trạng thái bật/tắt của từng thiết bị tại mỗi thời điểm. Quá trình này đòi hỏi mô hình có khả năng học và theo dõi sự biến đổi liên tục của tín hiệu trong thời gian thực.

\textbf{Đặc điểm nổi bật:} 

Phương pháp này cho phép mô hình học được đặc điểm tiêu thụ của mọi kịch bản bật–tắt thiết bị có thể xảy ra trong hộ gia đình, bao gồm cả các tình huống vận hành đơn lẻ, nhiều thiết bị thay đổi trạng thái đồng thời, các chế độ hoạt động khác nhau của từng thiết bị, cũng như các dạng tương tác phức tạp giữa các thiết bị đang hoạt động và các thiết bị có công suất thay đổi theo thời gian. Nhờ đó, mô hình có khả năng bao quát đầy đủ các kiểu hành vi tiêu thụ điện trong thực tế.

\textbf{Nhược điểm:} 

Phương pháp này tồn tại một hạn chế lớn: Mô hình phải học toàn bộ các tổ hợp bật/tắt của tất cả thiết bị trong hệ thống. Điều này khiến quá trình thu thập dữ liệu trở nên đặc biệt tốn thời gian và công sức, vì số lượng tổ hợp tăng theo cấp số nhân khi số thiết bị tăng lên.

\textbf{Ví dụ:}
Nếu hệ thống có 3 thiết bị gồm đèn (500W), máy lạnh (1000W) và bơm nước (1500W), 
tổng số tổ hợp trạng thái bật/tắt có thể xảy ra là \( 2^3 = 8 \).
Khi thêm thiết bị thứ tư, số tổ hợp tăng lên \( 2^4 = 16 \).
Điều này cho thấy việc mở rộng hệ thống sẽ khiến mô hình huấn luyện ngày càng 
phức tạp và khó khả thi trong thực tế.

\subsection{Hướng tiếp cận theo sự kiện (Event-based)}

\textbf{Nguyên lý hoạt động:}

Trong bối cảnh NILM, sự kiện được hiểu là thời điểm một thiết bị thay đổi trạng thái từ bật sang tắt hoặc từ tắt sang bật. Đây là những khoảnh khắc tạo ra biến động rõ rệt trong tín hiệu điện, đặc biệt là công suất tức thời.

Quy trình gồm ba bước chính. Trước hết, hệ thống thực hiện phát hiện thiết bị thay đổi trạng thái bằng cách xác định các thời điểm có sự thay đổi trạng thái của thiết bị dựa trên biến động công suất tức thời, thường được kiểm tra thông qua một ngưỡng công suất đặt trước. Những điểm có mức thay đổi lớn được xem như ứng viên cho sự kiện bật hoặc tắt.

Tiếp theo, với mỗi sự kiện được phát hiện, hệ thống tiến hành trích xuất đặc trưng bằng cách lấy một đoạn tín hiệu ngắn trước và sau thời điểm xảy ra sự kiện. Các đặc trưng thu được có thể bao gồm độ biến thiên công suất, gradient, dạng sóng tín hiệu hoặc các biến đổi liên quan khác phản ánh hành vi của thiết bị trong quá trình chuyển trạng thái.

Cuối cùng, mô hình máy học sử dụng các đặc trưng này để phân loại sự kiện, nhằm xác định loại thiết bị đã gây ra thao tác bật hoặc tắt đó.

\textbf{Ưu điểm của phương pháp:}

Phương pháp này chỉ tập trung vào các thời điểm thiết bị bật hoặc tắt, nhờ đó giảm đáng kể chi phí thu thập và chuẩn bị dữ liệu. Việc khai thác trực tiếp các thời điểm thay đổi trạng thái — nơi tập trung phần lớn thông tin quan trọng — giúp mô hình phân loại hoạt động hiệu quả hơn, đồng thời tránh được nhiễu xuất hiện trong những đoạn tín hiệu mà không có thiết bị nào thay đổi.

\subsection{So sánh hai hướng tiếp cận trong NILM}
\begin{table}[H]{So sánh giữa Continuous-based NILM và Event-based NILM}
    \centering
    \begin{tabularx}{\linewidth}{|X|X|X|}
        \hline
        \textbf{Tiêu chí} & \textbf{Continuous-based NILM} & \textbf{Event-based NILM} \\ \hline

        Cách tiếp cận 
        & Phân tích toàn bộ chuỗi tín hiệu theo thời gian 
        & Chỉ phân tích thời điểm thiết bị bật hoặc tắt\\ \hline

        Tín hiệu đầu vào 
        & Chuỗi tín hiệu liên tục: $P(t)$, $I(t)$, \ldots 
        & Đoạn tín hiệu ngắn quanh thời điểm sự kiện \\ \hline

        Yêu cầu dữ liệu gán nhãn 
        & Cần gán nhãn từng thời điểm 
        & Chỉ cần gán nhãn tại thời điểm bật/tắt \\ \hline

        Khả năng mở rộng 
        & Khó mở rộng nếu số lượng thiết bị lớn 
        & Dễ mở rộng vì chỉ xử lý các điểm quan trọng \\ \hline

        Phù hợp cho hệ thống thực tế 
        & Hệ thống có số lượng thiết bị giới hạn 
        & Hệ thống có nhiều thiết bị và ưu tiên theo dõi thời điểm bật/tắt \\ \hline

        Hạn chế chính 
        & Khó huấn luyện và khó mở rộng mô hình 
        & Có nguy cơ bỏ sót thiết bị hoặc nhầm lẫn với nhiễu \\ \hline
    \end{tabularx}
\end{table}

\textbf{Lưu ý khi lựa chọn hướng tiếp cận}

Mặc dù phương pháp dựa trên sự kiện có ưu thế rõ rệt về khả năng mở rộng và tính nhẹ, nó vẫn tiềm ẩn rủi ro tích lũy sai số nếu xuất hiện tình trạng bỏ sót hoặc phân loại sai sự kiện. Việc lựa chọn phương pháp phụ thuộc chặt chẽ vào mục tiêu của hệ thống. Nếu hệ thống chỉ cần theo dõi một số thiết bị cố định và ít thay đổi, phương pháp phân tích liên tục có thể phù hợp hơn vì nó mô tả chi tiết toàn bộ tín hiệu. Ngược lại, trong môi trường có nhiều thiết bị hoặc thường xuyên thay đổi, cách tiếp cận dựa trên sự kiện mang lại lợi thế nhờ khả năng mở rộng linh hoạt và giảm chi phí huấn luyện.


\section{Đặt vấn đề}

Trong những năm gần đây, cùng với xu hướng phát triển của lưới điện thông minh (Smart Grid) và nhu cầu nâng cao hiệu quả sử dụng năng lượng, bài toán phân tích tiêu thụ điện năng trong các hộ gia đình và tòa nhà dân dụng ngày càng nhận được nhiều sự quan tâm. Hệ thống giám sát tải không xâm nhập (Non-Intrusive Load Monitoring – NILM) được xem là một trong những giải pháp tiềm năng, cho phép xác định trạng thái hoạt động và mức tiêu thụ điện năng của từng thiết bị điện chỉ dựa trên tín hiệu đo tại một điểm duy nhất ở đầu nguồn, mà không cần lắp đặt thêm các công tơ riêng lẻ cho mỗi thiết bị.

Mặc dù đã được nghiên cứu trong nhiều năm và đạt được nhiều kết quả đáng kể trong các môi trường thử nghiệm, việc triển khai hệ thống NILM trong thực tế vẫn còn gặp phải nhiều thách thức lớn. Nhiều thiết bị điện hiện đại như máy điều hòa không khí sử dụng công nghệ inverter, tủ lạnh, máy giặt, laptop,hoặc các thiết bị điện tử không có mức công suất tiêu thụ cố định mà liên tục dao động theo chế độ vận hành và điều kiện môi trường. Điều này khiến cho các phương pháp NILM truyền thống dựa trên ngưỡng công suất trở nên kém hiệu quả. Ngoài ra, các yếu tố như nhiễu đo lường, dao động điện áp của lưới điện, sai số của cảm biến và sự thay đổi hành vi sử dụng điện của người dùng theo thời gian cũng góp phần làm cho dữ liệu đo trở nên không ổn định và khó xử lý hơn rất nhiều so với dữ liệu trong môi trường lý tưởng của phòng thí nghiệm \cite{ref2}.

Một hạn chế quan trọng khác của các nghiên cứu hiện tại là sự phụ thuộc lớn vào các bộ dữ liệu tiêu chuẩn được công bố rộng rãi như REDD, UK-DALE, PLAID hay REFIT. Mặc dù đây là những bộ dữ liệu có giá trị tham khảo cao, chúng chủ yếu được thu thập tại các quốc gia phát triển, với có tần số lấy mẫu cao, không có thiết bị hoạt động phức tạp, môi trường ổn định và có điều kiện lưới điện khác với Việt Nam. Vì vậy, việc áp dụng trực tiếp các mô hình được huấn luyện từ những bộ dữ liệu nước ngoài vào thực tế trong nước vì thế thường không đạt được độ chính xác như kỳ vọng.

Ngoài ra, trong điều kiện thực tế, các thuật toán nhận diện thiết bị thay đổi trạng thái có tỷ lệ bỏ lỡ sự kiện, mà một sự kiện nếu bị bỏ sót hoặc bị nhận dạng sai ngay từ ban đầu có thể dẫn đến sự sai lệch dây chuyền trong quá trình theo dõi trạng thái các thiết bị về sau, làm tích lũy sai số và gây ảnh hưởng nghiêm trọng đến độ tin cậy của toàn bộ hệ thống NILM. Điều này đặc biệt quan trọng trong các ứng dụng yêu cầu độ chính xác cao như quản lý năng lượng thông minh, chẩn đoán lỗi thiết bị hoặc hỗ trợ người dùng tối ưu hóa mức tiêu thụ điện năng.

Xuất phát từ những hạn chế và thách thức nêu trên, có thể thấy rằng việc xây dựng một thuật toán NILM có khả năng hoạt động tốt trong môi trường thực tế, với nhiều thiết bị hoạt động đồng thời, phức tạp và dựa trên dữ liệu thu thập tại môi trường trong nước là một vấn đề cần thiết và có ý nghĩa thực tiễn cao. Điều này đặt ra yêu cầu phải nghiên cứu sâu hơn về phương pháp phát hiện sự kiện, trích chọn đặc trưng và nhận dạng thiết bị dựa trên dữ liệu thực nghiệm, đồng thời giảm thiểu sự phụ thuộc vào những giả định lý tưởng thường được sử dụng trong các nghiên cứu trước đây.

\section{Mục tiêu nghiên cứu}

Đề tài được thực hiện với mục tiêu xây dựng, kiểm thử và đánh giá thuật toán NILM theo hướng sự kiện (Event-based NILM) có khả năng vận hành ổn định trong điều kiện thực tế, nơi tín hiệu điện thường bị ảnh hưởng bởi nhiễu và sự dao động của phụ tải. Nghiên cứu tập trung trước hết vào việc phát triển phương pháp phát hiện chính xác thời điểm bật hoặc tắt của thiết bị dựa trên tín hiệu điện tổng thu thập từ hệ thống thực, ngay cả khi nhiều thiết bị phức tạp hoạt động đồng thời và môi trường chứa nhiều nhiễu. 

Bên cạnh đó, đề tài hướng đến việc giảm thiểu tỷ lệ bỏ sót sự kiện và hạn chế tối đa các sự kiện giả thông qua việc đề xuất hoặc điều chỉnh các thuật toán xử lý tín hiệu, qua đó nâng cao độ ổn định và độ tin cậy của quá trình phát hiện sự kiện. Tiếp theo, nghiên cứu thực hiện nhận dạng thiết bị dựa trên các đặc trưng tín hiệu được trích xuất từ dữ liệu đo thực như công suất tức thời, dạng sóng dòng điện, điện áp nhằm tăng độ chính xác của quá trình phân loại trong điều kiện tín hiệu thực nghiệm phức tạp.

Một đóng góp quan trọng của đề tài là xây dựng và công bố một bộ dữ liệu thực nghiệm mới, bao gồm tín hiệu điện áp và dòng điện tức thời của nhiều thiết bị gia dụng trong các sự kiện bật và tắt. Bộ dữ liệu được thu thập trong điều kiện lưới điện Việt Nam, tạo ra một nền tảng đánh giá phù hợp cho các phương pháp NILM hướng sự kiện và góp phần khắc phục sự thiếu hụt dữ liệu bản địa.

Toàn bộ thuật toán được triển khai, mô phỏng và kiểm thử trong môi trường Python theo hướng xử lý ngoại tuyến, bao gồm trực quan hóa tín hiệu, phân tích đặc trưng và đánh giá hiệu năng để kiểm chứng tính khả thi và độ chính xác của hệ thống. Cuối cùng, nghiên cứu đề xuất các hướng phát triển ứng dụng thực tế, mở ra khả năng triển khai hệ thống NILM trong hộ gia đình hoặc các mô hình quản lý năng lượng thông minh trong tương lai, từ đó góp phần nâng cao hiệu quả sử dụng năng lượng và cho phép người dùng theo dõi phụ tải chi tiết mà không cần trang bị cảm biến riêng lẻ cho từng thiết bị.

\section{Cấu trúc của đồ án}

Đồ án này được tổ chức thành 8 chương, mỗi chương trình bày một cách mạch lạc và chi tiết, giúp người đọc dễ dàng theo dõi quá trình nghiên cứu, từ lý thuyết cơ bản đến triển khai và đánh giá thuật toán NILM dựa trên dữ liệu đo thực tế.

Cấu trúc cụ thể của đồ án như sau:

\textbf{Chương 1: Giới thiệu} – Trình bày bối cảnh nghiên cứu về NILM, các hướng tiếp cận phổ biến, đặt vấn đề và mục tiêu nghiên cứu. Đồng thời, chương này nêu tổng quan về đóng góp chính của đề tài, bao gồm xây dựng thuật toán NILM hướng sự kiện và bộ dữ liệu thực nghiệm mới.
    
\textbf{Chương 2: Tổng quan thuật toán NILM đề xuất} – Mô tả kiến trúc và các thành phần chính của thuật toán, bao gồm khối phát hiện sự kiện, tiền xử lý và trích xuất đặc trưng, nhận dạng thiết bị.
    
\textbf{Chương 3: Bộ đo thu thập dữ liệu và dữ liệu sử dụng} – Trình bày về bộ đo được sử dụng, phương pháp thu thập dữ liệu thực nghiệm, bao gồm các sự kiện bật/tắt của nhiều thiết bị, và thông tin tập dữ liệu được sử dụng.
    
\textbf{Chương 4: Thuật toán phát hiện sự kiện} – Trình bày tổng quan thuật toán phát hiện sự kiện đề xuất được áp dụng, áp dụng nhiều phương pháp bao gồm lọc trung bình, WAMMA, bộ lọc Kalman, và thuật toán Hybrid của Mengqi Lu và Zuyi Li. 
    
\textbf{Chương 5: Tiền xử lý dữ liệu và trích xuất đặc trưng} – Trình bày mục tiêu của giai đoạn tiền xử lý, phương pháp trích xuất đặc trưng công suất và đặc trưng ảnh, cơ sở lý thuyết, vấn đề thực tế, phương pháp tính toán và kết quả thu được.
    
\textbf{Chương 6: Mô hình học máy} – Trình bày các mô hình học máy được áp dụng cho việc nhận dạng thiết bị dựa trên đặc trưng tín hiệu đã trích xuất, cũng như các bước huấn luyện, kiểm thử và tối ưu hóa mô hình.
    
\textbf{Chương 7: Kết quả và đánh giá thuật toán} – Trình bày dữ liệu đánh giá, quy trình đánh giá, kết quả thực nghiệm của các thuật toán phát hiện sự kiện, nhận xét và phân tích hiệu quả thuật toán.

\textbf{Chương 8: Hương nghiên cứu tương lai} – Trình bày các định hướng mở rộng của hệ thống NILM, bao gồm chuyển từ xử lý ngoại tuyến sang vận hành thời gian thực, giải quyết hạn chế khi số lượng thiết bị lớn và đề xuất hướng tiếp cận phân nhóm thiết bị theo công suất để nâng cao độ chính xác và khả năng mở rộng của mô hình.

Cấu trúc này giúp làm rõ quá trình nghiên cứu từ lý thuyết đến triển khai thực nghiệm và đánh giá thuật toán, đồng thời nhấn mạnh đóng góp chính của đề tài trong việc phát triển thuật toán NILM hướng sự kiện và bộ dữ liệu thực nghiệm phục vụ quản lý năng lượng hiệu quả trong môi trường gia đình.

\chapter{Tổng quan thuật toán NILM đề xuất}

Thuật toán NILM đề xuất trong đồ án được thiết kế theo hướng mô-đun hoá, cho phép
tách biệt rõ ràng vai trò của từng khối xử lý và đảm bảo tính mở rộng khi triển khai thực
tế. Kiến trúc tổng thể gồm ba thành phần chính: (1) Khối phát hiện sự kiện bật/tắt, (2) Khối tiền xử lý và trích xuất đặc trưng, và (3) Khối nhận diện thiết
bị. Các khối này phối hợp theo chu trình xử lý tuần tự, từ việc thu nhận dữ liệu thô đến 
việc xác định chính xác danh tính thiết bị gây ra sự kiện tiêu thụ điện \cite{ref3}.

\begin{figure}[H]{Kiến trúc tổng thể của thuật toán NILM đề xuất}
    \centering
    \includegraphics[width=0.9\linewidth]{IMAGE/LuongHeThongNILM.jpg}
\end{figure}

\section{Khối phát hiện sự kiện bật/tắt thiết bị}

Khối phát hiện sự kiện có vai trò xác định chính xác thời điểm một thiết bị thay đổi trạng thái hoạt động, tức chuyển từ bật sang tắt hoặc ngược lại. Đây là một trong những thành phần quan trọng nhất của toàn bộ thuật toán NILM, bởi chỉ cần giai đoạn này sai lệch, mọi bước xử lý phía sau — bao gồm trích xuất đặc trưng và nhận dạng thiết bị — đều bị ảnh hưởng và kéo theo sự suy giảm đáng kể của độ chính xác toàn thuật toán. Do đó, việc xây dựng một mô-đun phát hiện sự kiện ổn định, nhạy và đáng tin cậy là yêu cầu then chốt của thuật toán.

Trong mô hình đề xuất của đề tài, khối phát hiện sự kiện được thiết kế theo hướng kết hợp nhiều kỹ thuật xử lý tín hiệu khác nhau. Các tín hiệu công suất ban đầu được làm mượt để giảm thiểu nhiễu sinh ra từ dao động tức thời của các phụ tải. Tiếp theo, thuật toán sẽ tăng cường độ nhạy với các thay đổi công suất nhỏ nhằm đảm bảo rằng cả những thiết bị có công suất thấp hoặc chuyển trạng thái nhanh cũng được nhận diện đầy đủ. Bên cạnh đó, mô hình còn chú trọng cải thiện khả năng nhận diện sự kiện trong những tình huống có nhiều thiết bị hoạt động đồng thời — một bài toán khó thường gặp trong môi trường gia dụng thực tế, nơi các tín hiệu dễ chồng lấn và gây nhầm lẫn trong quá trình phân tích.

Kết quả đầu ra của khối này là danh sách các thời điểm xảy ra sự kiện bật hoặc tắt của thiết bị, kèm theo các thông tin đặc trưng liên quan đến biến động tín hiệu tại thời điểm đó. Những thời điểm này đóng vai trò làm điểm neo (anchor points) cho quá trình trích xuất đặc trưng ở bước tiếp theo, bảo đảm hệ thống chỉ phân tích những đoạn tín hiệu thực sự quan trọng thay vì toàn bộ chuỗi dữ liệu liên tục. Cách tiếp cận này không chỉ giúp giảm đáng kể chi phí tính toán mà còn cải thiện độ chính xác tổng thể của thuật toán, đặc biệt trong điều kiện tín hiệu thực nghiệm phức tạp và nhiễu cao.

\section{Khối tiền xử lý và trích xuất đặc trưng}

Sau khi các sự kiện bật hoặc tắt được phát hiện, thuật toán tiến hành cắt và trích đoạn tín hiệu trong một cửa sổ thời gian bao quanh thời điểm sự kiện. Việc lựa chọn cửa sổ thời gian phù hợp giúp đảm bảo rằng cả tín hiệu trước và sau sự kiện đều được ghi nhận đầy đủ, từ đó cung cấp nguồn dữ liệu giàu thông tin cho quá trình phân tích. Từ các đoạn tín hiệu này, thuật toán tiến hành rút trích nhiều dạng đặc trưng khác nhau nhằm biểu diễn hành vi của thiết bị khi xảy ra sự chuyển trạng thái.

Các đặc trưng được trích xuất bao gồm độ thay đổi công suất $\Delta P$ trước và sau sự kiện, vốn phản ánh mức tiêu thụ năng lượng tức thời của thiết bị khi chuyển trạng thái. Bên cạnh đó, dạng sóng I--V cũng được phân tích nhằm khai thác các dạng biến thiên tín hiệu mang tính đặc thù của từng loại thiết bị. Ngoài các đặc trưng trực tiếp từ miền thời gian, thuật toán còn chuyển đổi và chuẩn hoá các đoạn tín hiệu sang dạng ảnh, tạo ra một biểu diễn trực quan phù hợp với các mô hình học máy và đặc biệt hữu ích cho các thuật toán học sâu.

Khối trích xuất đặc trưng đóng vai trò như cầu nối giữa xử lý tín hiệu truyền thống và các mô hình phân loại. Việc chuyển đổi dữ liệu thô sang một không gian đặc trưng giàu thông tin và ổn định giúp tăng đáng kể hiệu quả của bước nhận dạng thiết bị. Nhờ đó, thuật toán có thể khai thác đầy đủ các thông tin quan trọng trong tín hiệu, đồng thời giảm thiểu nhiễu và các yếu tố gây sai số xuất hiện trong quá trình đo đạc thực nghiệm.

\section{Khối nhận diện thiết bị}

Dựa trên các đặc trưng được trích xuất, thuật toán sử dụng mô hình học máy MLP (Multilayer Perceptron) để nhận diện thiết bị gây ra sự kiện. Mô hình được huấn luyện 
từ các đặc trưng ảnh tín hiệu và các đặc trưng công suất, cho phép phân biệt các thiết bị có đặc trưung tiêu thụ tương đối gần nhau. Đầu ra của khối này là nhãn thiết bị tương ứng với từng sự kiện bật/tắt, cung cấp khả năng theo dõi hoạt động thiết bị theo thời gian thực.


\chapter{Bộ đo thu thập dữ liệu và dữ liệu sử dụng}

Trong phạm vi đồ án, dữ liệu điện áp (U) và dòng điện (I) tức thời của hệ thống điện gia đình được thu thập với mục tiêu tạo ra tập dữ liệu phục vụ bài toán phát hiện và nhận diện sự kiện trong giám sát tải không xâm lấn (NILM). Việc thu thập được tiến hành trong môi trường có kiểm soát, nơi các sự kiện bật/tắt thiết bị được tạo ra một cách chủ động để đảm bảo dữ liệu có chất lượng và độ rõ ràng cao, phù hợp cho quá trình đánh giá thuật toán.


\section{Cấu hình hệ thống đo}

Để đảm bảo thu nhận được tín hiệu tức thời với độ chính xác cao, hệ thống đo được thiết kế dựa trên một tập hợp phần cứng phù hợp với điều kiện điện lưới Việt Nam cũng như yêu cầu của bài toán NILM theo hướng sự kiện. Các thành phần được lựa chọn không chỉ đảm bảo khả năng đo đúng biên dạng tín hiệu, mà còn phải chịu được nhiễu, đáp ứng đủ nhanh và vận hành an toàn trong môi trường gia dụng.

\textbf{Điều kiện điện lưới Việt Nam}

Điện áp danh định của lưới điện Việt Nam là 220~V AC với tần số 50~Hz. Hai thông số cơ bản này có ảnh hưởng trực tiếp đến cấu trúc tín hiệu cần thu thập, tốc độ lấy mẫu tối thiểu, dải hoạt động của cảm biến và khả năng phát hiện các thành phần hài hoặc nhiễu cao tần. Việc nắm rõ đặc điểm lưới điện thực tế là cơ sở quan trọng để lựa chọn phần cứng đo lường phù hợp và thiết kế bộ lọc, thuật toán trích xuất đặc trưng.

\textbf{Cảm biến dòng SCT013}

Cảm biến SCT013 được sử dụng như một giải pháp đo dòng không xâm lấn, giúp đảm bảo an toàn tuyệt đối vì hoàn toàn cách ly với lưới điện. Thiết bị này có thể kẹp vào bất kỳ dây tải nào mà không cần can thiệp vào kết nối điện, từ đó hỗ trợ quá trình lắp đặt nhanh chóng và phù hợp với nhiều thiết bị gia dụng. Trong khoảng dòng từ 0--20~A, SCT013 cho độ tuyến tính đủ tốt đối với các ứng dụng NILM, đồng thời có khả năng ghi nhận chính xác các dòng khởi động lớn, vốn là đặc trưng quan trọng khi phân tách và nhận dạng thiết bị.

\textbf{Cảm biến điện áp ZMPT107}

Việc đo dạng sóng điện áp đóng vai trò quan trọng trong bài toán NILM, bởi nó cung cấp thêm thông tin về biến dạng sóng, sự lệch pha giữa điện áp và dòng điện cũng như các nhiễu chuyển tiếp xảy ra trong thời gian rất ngắn (dưới 20~ms). Cảm biến ZMPT107 đáp ứng tốt các yêu cầu này nhờ độ chính xác cao trong dải điện áp AC 0--250~V và thiết kế tích hợp biến áp cách ly giúp đảm bảo an toàn khi kết nối với lưới điện. Tốc độ đáp ứng của cảm biến đủ nhanh để tái tạo chính xác dạng sóng, phục vụ quá trình phân tích sự kiện bật/tắt của thiết bị.

\textbf{IC đo điện năng BL0940}

BL0940 \cite{ref4} là bộ xử lý năng lượng chuyên dụng được lựa chọn làm lõi thu thập dữ liệu. So với các IC như ADE7755 hay HLW8012, BL0940 có độ phân giải ADC cao hơn và khả năng xuất dữ liệu tức thời thay vì chỉ cung cấp các giá trị RMS, điều này đặc biệt quan trọng đối với bài toán phát hiện sự kiện. Ngoài ra, BL0940 tích hợp các bộ lọc chống nhiễu tần số cao, giúp tín hiệu thu thập ổn định hơn trong điều kiện môi trường nhiều nhiễu như lưới điện dân dụng. Giao tiếp SPI tốc độ cao của IC hỗ trợ truyền dữ liệu liên tục mà không mất mẫu. BL0940 có thể cung cấp đồng thời điện áp tức thời, dòng điện tức thời, công suất tức thời và các giá trị RMS theo cửa sổ trượt, nhờ đó phù hợp cho cả các bài toán NILM theo sự kiện và NILM theo dạng sóng.

\textbf{Raspberry Pi 4 làm thiết bị thu nhận dữ liệu}

Raspberry Pi~4 được sử dụng làm bộ xử lý trung tâm trong quá trình thu thập tín hiệu. Thiết bị này đảm nhận nhiệm vụ nhận dữ liệu tốc độ cao từ BL0940 thông qua giao tiếp SPI, ghi tín hiệu theo thời gian thực vào các tệp CSV và gán timestamp có độ chính xác đến microsecond. Với khả năng chạy Python linh hoạt, Pi~4 cho phép tích hợp trực tiếp các đoạn mã xử lý dữ liệu thô hoặc thực hiện tiền xử lý ngay tại thời điểm thu nhận. Nhờ năng lực tính toán ổn định và bộ nhớ đủ lớn, hệ thống đảm bảo không xảy ra hiện tượng mất mẫu, đồng thời tạo nền tảng vững chắc cho các bước phân tích tiếp theo.

Tổng thể, cấu hình phần cứng được lựa chọn dựa trên sự cân bằng giữa độ chính xác đo lường, tính an toàn, chi phí hợp lý và khả năng đáp ứng yêu cầu của bài toán NILM trên tín hiệu thực nghiệm. Bộ thiết bị này cho phép tái tạo dạng sóng với độ trung thực cao, từ đó hỗ trợ hiệu quả các thuật toán phát hiện sự kiện và nhận dạng thiết bị.


\begin{figure}[H]{Sơ đồ hệ thống thu thập dữ liệu được sử dụng}
    \centering
    \includegraphics[width=0.9\linewidth]{IMAGE/measurement_system.png}
\end{figure}

\section{Phương pháp thu thập}

Quá trình thu thập dữ liệu được thực hiện bằng cách ghi nhận liên tục tín hiệu điện áp tức thời \(U\) và dòng điện tức thời \(I\) trong khi hệ thống hoạt động với nhiều trạng thái thiết bị khác nhau. Ở thời điểm bắt đầu mỗi phiên đo, một nhóm thiết bị được lựa chọn và để ở trạng thái ổn định, chẳng hạn như đèn được bật, quạt đang chạy hoặc máy tính xách tay đang trong quá trình sạc. Trong điều kiện vận hành này, hệ thống thu thập tín hiệu tức thời \(U\) và \(I\) một cách liên tục theo thời gian, phản ánh chính xác hoạt động thực tế của tải điện trong môi trường đo.

Trong khoảng hai phút đầu tiên, tất cả thiết bị được giữ nguyên nhằm tạo ra đoạn dữ liệu nền ổn định. Sau đó, trạng thái của một thiết bị duy nhất được thay đổi có chủ đích, ví dụ như bật thêm tủ lạnh, tắt bộ sạc máy tính. Hệ thống tiếp tục duy trì việc ghi nhận tín hiệu thêm khoảng hai phút để bảo đảm dữ liệu trước và sau sự kiện được phân tách rõ ràng. Quy trình này được lặp lại nhiều lần, mỗi lần thay đổi trạng thái của một thiết bị khác. Trong vòng khoảng mười phút, phiên đo sẽ ghi nhận được một chuỗi dữ liệu liên tục chứa nhiều lần thay đổi trạng thái thiết bị, tạo ra một tệp dữ liệu duy nhất lưu trữ toàn bộ tín hiệu \(U\) và \(I\) tức thời của hệ thống điện trong suốt giai đoạn đó.

Mỗi tệp thường chứa bốn đến năm sự kiện tương ứng với bốn đến năm thiết bị được thay đổi trạng thái theo thứ tự. Để thuận lợi cho quá trình phân tích, gán nhãn và đánh giá thuật toán phát hiện sự kiện, chúng tôi chủ động tách tệp dữ liệu ban đầu thành nhiều tệp nhỏ hơn, mỗi tệp tương ứng với đúng một thiết bị thay đổi trạng thái. Việc tách này giúp cô lập từng sự kiện một cách rõ ràng, giảm nhiễu chéo giữa các lần chuyển trạng thái và tăng độ tin cậy cho quá trình huấn luyện và kiểm thử các thuật toán NILM. Điều này cũng bảo đảm rằng mỗi tệp dữ liệu con thể hiện trọn vẹn một chuỗi ``trước -- trong -- sau'' sự kiện của duy nhất một thiết bị, giúp phương pháp phân tích đạt độ chính xác cao hơn.

\section{Mô tả tập dữ liệu}

Tập dữ liệu được xây dựng từ quá trình đo đạc tín hiệu điện áp tức thời \(U\) và dòng điện tức thời \(I\) của năm thiết bị gia dụng phổ biến, bao gồm sạc máy tính, quạt điện, tủ lạnh, máy sấy tóc và máy ép trái cây. Công suất trung bình tương ứng của các thiết bị này lần lượt vào khoảng 80\,W, 40\,W, 85\,W, 330\,W và 30\,W. Các thiết bị được lựa chọn nhằm đảm bảo sự đa dạng về đặc tính phụ tải, từ thiết bị tuyến tính đơn giản (như quạt) đến thiết bị phi tuyến (như sạc máy tính), giúp phản ánh đúng các dạng phụ tải thường gặp trong môi trường gia đình.

Như đã mô tả trong phần thu thập dữ liệu, mỗi phiên đo kéo dài khoảng mười phút và bao gồm nhiều lần thay đổi trạng thái thiết bị. Sau khi hoàn tất thu thập, mỗi phiên được tách thành nhiều tệp nhỏ hơn, trong đó mỗi tệp chỉ chứa duy nhất một sự kiện thay đổi trạng thái của một thiết bị. Các tệp này được tạo trong nhiều điều kiện kết hợp khác nhau, chẳng hạn một thiết bị bật trong khi các thiết bị còn lại tắt, hoặc một thiết bị thay đổi trạng thái trong khi có hai hoặc ba thiết bị khác đang hoạt động. Điều này cho phép xây dựng một tập dữ liệu phản ánh chân thực môi trường vận hành thực tế, nơi các thiết bị không hoạt động độc lập mà luôn ảnh hưởng lẫn nhau.

Sau quá trình chia tách, tổng số tệp dữ liệu thu được là 116 tệp, mỗi tệp tương ứng với một lần thay đổi trạng thái của một trong năm thiết bị. Ngoài các tệp chứa sự kiện, tập dữ liệu còn bao gồm các tệp riêng biệt trong đó từng thiết bị được bật một mình trong khoảng hai phút. Các tệp này không chứa sự kiện chuyển trạng thái mà chỉ phản ánh đặc trưng vận hành ổn định của từng thiết bị. Chúng được sử dụng làm dữ liệu huấn luyện cho các mô hình học máy ở các giai đoạn sau, đặc biệt trong việc xây dựng đặc trưng nhận diện và phân biệt từng thiết bị.

Với cấu trúc như vậy, tập dữ liệu vừa đảm bảo tính đa dạng của môi trường vận hành thực tế, vừa cung cấp các mẫu sạch dùng cho học máy, cho phép đánh giá và huấn luyện các thuật toán phát hiện sự kiện và nhận diện thiết bị trong bối cảnh bài toán NILM.


\chapter{Thuật toán phát hiện sự kiện}

\section{Tổng quan thuật toán phát hiện sự kiện}

Trong các hệ thống NILM hiện đại, phát hiện sự kiện (event detection) là bước quan trọng đầu tiên nhằm xác định thời điểm một thiết bị điện thay đổi trạng thái (bật hoặc tắt). Chất lượng của khối phát hiện sự kiện ảnh hưởng trực tiếp đến toàn bộ quá trình nhận diện thiết bị thay đổi trạng thái, bởi mọi bài toán con về sau đều phụ thuộc vào việc sự kiện được xác định chính xác.

Phần lớn dữ liệu đo tại hộ gia đình thường bị nhiễu, không ổn định và chứa nhiều dao động nhỏ đặc biết khi có sự xuất hiện của thiết bị có công suất không ổn định. Vì vậy, thuật toán phát hiện sự kiện cần xử lý đồng thời hai yêu cầu: (1) Duy trì độ nhạy đủ lớn để không bỏ sót sự kiện nhỏ, (2) Hạn chế sinh nhiễu báo động giả (false positive) do thiết bị có công suất thay đổi mạnh theo thời gian.

Trong đề tài này, thuật toán phát hiện sự kiện được xây dựng dựa trên nhiều thuật toán phát hiện sự kiện và các bộ lọc để giảm nhiễu, kết hợp các kỹ thuật xử lý tín hiệu thời gian thực để tận dụng đặc tính của từng loại thuật toán.

\textbf{Một số thuật ngữ cơ bản:}
\begin{table}[h]{Giải thích thuật ngữ về thuật toán phát hiện sự kiện}
    \centering
    \begin{adjustbox}{width=\linewidth}
        \begin{tabularx}{\linewidth}{| >{\centering\arraybackslash}p{3cm} | X |}
            \hline
            \textbf{Thuật ngữ} & \textbf{Giải thích} \\
            \hline
            Cửa sổ (Window) & Một đoạn tín hiệu con được trích từ chuỗi tín hiệu dài để phân tích tại một thời điểm xác định. \\
            \hline
            CVD & Thiết bị có công suất biến đổi mạnh theo thời gian. \\
            \hline
        \end{tabularx}
    \end{adjustbox}
\end{table}

\textbf{Nguyên lý hoạt động}

Tín hiệu công suất P được chia thành các cửa sổ thời gian có độ dài xác định. Với mỗi cửa sổ trượt, thuật toán so sánh công suất trung bình giữa hai phía (trước và sau) quanh một điểm nghi ngờ là sự kiện:
$$
\Delta P = \left| \overline{P}_{\text{trước}} - \overline{P}_{\text{sau}} \right|
$$

Nếu $\Delta P$ vượt quá một ngưỡng định trước, điểm này được xem là có khả năng sự kiện xảy ra — tức là thiết bị đã thay đổi trạng thái.

Quy trình tổng thể của hệ thống phát hiện sự kiện được mô tả như sau:

\textbf{1. Lọc và giảm tần số mẫu đầu vào} 

Dữ liệu ban đầu gồm điện áp và dòng điện được lấy mẫu ở tần số cao. Tuy nhiên, tần số mẫu lớn không phải lúc nào cũng cần thiết cho nhiệm vụ phát hiện sự kiện, và có thể làm tăng nhiễu do dao động tức thời. Do đó, giai đoạn đầu tiên thực hiện giảm tần số mẫu (downsampling) nhằm loại bỏ nhiễu tần số cao và chuẩn hóa dữ liệu. Bước này vừa giảm khối lượng tính toán, vừa giúp tín hiệu ổn định hơn trước khi đưa vào các thuật toán phát hiện biên.

\textbf{2. Tách tín hiệu thành hai nhánh xử lý song song}

\textit{Nhánh 1 – Thuật toán phát hiện tần số cao:}

Ở nhánh này, đề tài sử dụng thuật toán WAMMA (Window with Adaptive Margins and Multi-window Analysis) làm phương pháp phát hiện sự kiện chính. Thuật toán WAMMA hoạt động hiệu quả với các thiết bị có quá trình chuyển trạng thái chậm. Tuy nhiên, nó gặp khó khăn khi môi trường có thiết bị CVDs, khiến việc phát hiện các thiết bị chuyển trạng thái trong môi trường đó trở nên kém chính xác.

\textit{Nhánh 2 – Thuật toán phát hiện tần số thấp:}

Để bù đắp hạn chế của nhánh tần số cao, đề tài bổ sung thêm một nhánh phân tích tần số thấp nhằm cải thiện khả năng phát hiện sự kiện trong môi trường có thiết bị CVD đang hoạt động. Nhánh này sử dụng một phiên bản cải tiến của thuật toán MLZLI (Hybrid của Mengqi Lu và Zuyi Li).

Trong nhánh này, tín hiệu được chạy qua bộ lọc Kalman để làm mượt (smoothing) và loại bỏ nhiễu do thay đổi tức thời, giúp nhận biết chính xác hơn đâu là sự thay đổi thực sự của thiết bị. Kết quả của nhánh thấp tần có độ tin cậy cao nhưng phản ứng chậm.

\textbf{3. Khối gộp sự kiện và ra quyết định} 

Hai nhánh xử lý được kết hợp để tận dụng ưu điểm của cả hai:

Nhánh tần số cao: Phát hiện nhanh nhưng dễ bỏ lỡ thiết bị trong môi trường nhiễu mạnh.

Nhánh tần số thấp: Phát hiện chậm nhưng chính xác, ổn định.

Do thuật toán có 2 nhánh phát hiện sự kiện nên các sự kiện liên tiếp trong một khoảng thời gian ngắn được nhóm lại thành một sự kiện duy nhất. Điều này giúp giảm số lượng sự kiện giả và đảm bảo mỗi lần thay đổi trạng thái chỉ tạo ra một sự kiện duy nhất

\textbf{Ý nghĩa và vai trò của khối phát hiện sự kiện}

Với thuật toán nhiều tầng như trên, khối phát hiện sự kiện trong đề tài vừa đảm bảo hoạt động tốt trong nhiễu, vừa duy trì độ phản ứng nhanh cần thiết. Kết quả là toàn bộ luồng dữ liệu của thuật toán NILM trở nên ổn định hơn. Các sự kiện được xác định rõ ràng, ít bị bỏ lỡ và phù hợp cho giai đoạn phân loại thiết bị phía sau.

\begin{figure}[H]{Sơ đồ thuật toán phát hiện sự kiện đề xuất}
    \centering
    \includegraphics[width=0.9\linewidth]{IMAGE/Event_Dectection/sodoevt.jpg}
\end{figure}

\section{Lọc trung bình}

Bộ lọc trung bình (Moving Average Filter) là một trong những bộ lọc số đơn giản và phổ biến nhất trong xử lý tín hiệu rời rạc. Mục đích chính của bộ lọc là làm mượt tín hiệu bằng cách thay thế giá trị hiện tại bằng giá trị trung bình của một nhóm các mẫu lân cận. Phương pháp này đặc biệt hiệu quả trong việc loại bỏ nhiễu ngẫu nhiên biên độ nhỏ, giúp tín hiệu trở nên ổn định hơn trước khi đưa vào các thuật toán phân tích sự kiện trong bài toán NILM.

\textbf{Cơ sở toán học của bộ lọc}

Bộ lọc trung bình cửa sổ \(M\) điểm được biểu diễn theo công thức:
\[
y[n] = \frac{1}{M} \sum_{k=0}^{M-1} x[n-k]
\]
Trong đó:
\begin{itemize}
    \item \(x[n]\): mẫu tín hiệu đầu vào tại thời điểm \(n\),
    \item \(y[n]\): mẫu tín hiệu sau khi lọc,
    \item \(M\): kích thước cửa sổ trung bình.
\end{itemize}

\section{Thuật toán WAMMA}

\subsection{Giới thiệu}

Trong bước phát hiện sự kiện của hệ thống NILM, việc xác định chính xác các thời điểm thay đổi công suất là rất quan trọng để tách biệt các thiết bị điện khác nhau. Tuy nhiên, các phương pháp sử dụng tham số cố định thường bị ảnh hưởng bởi nhiễu, dao động hoặc các sự kiện xảy ra gần nhau. 

Thuật toán WAMMA (Window with Adaptive Margins and Multi-window Analysis) được đề xuất nhằm khắc phục các hạn chế này bằng cách sử dụng cơ chế tự thích ứng ở cả ngưỡng phát hiện và cấu trúc cửa sổ phân tích. Phương pháp này được trình bày trong nghiên cứu của Yan et al \cite{ref5} và đã chứng minh hiệu quả vượt trội trong việc phát hiện sự kiện công suất với độ chính xác cao.

\subsection{Nguyên lý hoạt động của thuật toán}

Thuật toán WAMMA dựa trên ba cơ chế chính:

\textbf{Adaptive Window Margins:} 
Cửa sổ trượt được chia thành ba vùng: vùng biên trái (Left Margin), vùng trung tâm và vùng biên phải (Right Margin). Hai biên này đại diện cho các trạng thái công suất trước và sau sự kiện. Trong quá trình phân tích, kích thước biên được điều chỉnh tự động dựa trên biến thiên công suất và xu hướng tín hiệu để đảm bảo vùng chuyển tiếp của sự kiện được bao trọn hoàn toàn. 
Cơ chế này giúp thuật toán thích nghi với độ dài khác nhau của các sự kiện chuyển tiếp (ví dụ: bật/tắt thiết bị công suất lớn thường có quá trình chuyển tiếp dài hơn so với thiết bị nhỏ).

\textbf{Adaptive Thresholding:}
Ngưỡng phát hiện sự kiện không cố định mà được cập nhật động theo độ lệch chuẩn của dữ liệu trong cửa sổ hiện tại. 
Cụ thể, ngưỡng thích ứng được tính theo công thức:
\[
P_{thr,w} = \max(P_{thr}, r_{thr} \times \sigma)
\]
trong đó $\sigma$ là độ lệch chuẩn của tín hiệu công suất trong cửa sổ, $P_{thr}$ là ngưỡng tối thiểu định trước, và $r_{thr}$ là hệ số tỉ lệ. 
Cơ chế này giúp thuật toán tự động điều chỉnh độ nhạy, giảm phát hiện giả khi tín hiệu nhiễu cao và tăng độ chính xác khi tín hiệu ổn định.

\textbf{Multi-window Screening:}
Để đảm bảo phát hiện ổn định, WAMMA có thể triển khai đồng thời nhiều cửa sổ có kích thước khác nhau. Nếu một sự kiện được phát hiện trong nhiều cửa sổ trượt liên tiếp hoặc ở các độ rộng cửa sổ khác nhau, kết quả đó được xác nhận là sự kiện thật. Cơ chế sàng lọc đa cửa sổ này giúp loại bỏ nhiễu ngắn hạn và cải thiện độ tin cậy của hệ thống.

Ngoài ra, WAMMA còn tích hợp cơ chế \textit{cross-validation} giữa các bộ dữ liệu để đánh giá tính ổn định của tham số, chứng minh khả năng tổng quát hóa cao trong môi trường thực tế.

\subsection{Mã giả thuật toán (theo triển khai trong đề tài)}

Thuật toán được mô tả ngắn gọn bằng mã giả dưới đây, dựa trên phần cài đặt Python của đề tài:

\begin{algorithm}[H]
\caption{Thuật toán WAMMA}
\begin{algorithmic}[1]
\Require Chuỗi công suất $P_t$, các tham số $N_w$ (Độ rộng cửa sổ), $N_m$ (Độ rộng biên), $P_{thr}$ (Ngưỡng công suất sự kiện), $r_{thr}$(Hệ số tỷ lệ thích ứng)
\State Khởi tạo cửa sổ trượt rỗng và giá trị biên ban đầu
\For{mỗi mẫu công suất $p$ trong chuỗi $P_t$}
    \State Thêm $p$ vào cửa sổ hiện tại
    \If{độ dài cửa sổ $<$ $N_w$}
        \State Tiếp tục đọc dữ liệu (chưa đủ mẫu)
        \State \textbf{continue}
    \EndIf
    \State Tính giá trị trung bình $u$ và độ lệch chuẩn $\sigma$
    \State Cập nhật ngưỡng thích ứng: $P_{thr,w} = \max(P_{thr}, r_{thr} \times \sigma)$
    \State Tính biến thiên công suất biên trái $\Delta P_L$ và biên phải $\Delta P_R$
    \If{$\Delta P_L$ hoặc $\Delta P_R$ vượt ngưỡng cục bộ}
        \State Điều chỉnh biên trái/phải, mở rộng cửa sổ
        \State \textbf{continue}
    \EndIf
    \State Tính trung bình biên trái $u_L$ và phải $u_R$
    \State $\Delta P = u_R - u_L$
    \If{$\Delta P > P_{thr,w}$}
        \State Ghi nhận sự kiện bật (ON)
    \ElsIf{$\Delta P < -P_{thr,w}$}
        \State Ghi nhận sự kiện tắt (OFF)
    \EndIf
    \State Khởi tạo lại cửa sổ để tiếp tục phát hiện sự kiện tiếp theo
\EndFor
\end{algorithmic}
\end{algorithm}

\section{Bộ lọc Kalman}
Bộ lọc Kalman \cite{ref6} là một phương pháp ước lượng trạng thái tối ưu, được sử dụng để làm mượt tín hiệu trong các hệ thống có nhiễu. Nguyên tắc cơ bản là kết hợp giá trị dự đoán từ dữ liệu trước đó với giá trị quan sát thực tế, nhằm tạo ra một tín hiệu mượt và chính xác hơn.

Với dữ liệu công suất 1 chiều, bộ lọc Kalman có thể được mô tả đơn giản như sau:

\begin{enumerate}
    \item \textbf{Dự đoán giá trị tiếp theo:}
    \[
    \hat{x}_{k|k-1} = \hat{x}_{k-1}
    \]

    \item \textbf{Cập nhật ước lượng với quan sát thực tế:}
    \[
    \hat{x}_k = \hat{x}_{k|k-1} + K_k (z_k - \hat{x}_{k|k-1})
    \]

    \item \textbf{Trọng số Kalman \(K_k\)} xác định mức độ tin tưởng vào quan sát mới, giá trị trong khoảng từ 0 đến 1.
\end{enumerate}

Trong đó: 
\begin{itemize}
    \item $\hat{x}_k$ là giá trị công suất đã làm mượt tại thời điểm $k$,
    \item $z_k$ là giá trị quan sát thực tế,
    \item $K_k$ là trọng số Kalman.
\end{itemize}

\textbf{Tác dụng trong đề tài:}  
Trong đồ án này, bộ lọc Kalman được áp dụng trước thuật toán Hybrid để làm mượt tín hiệu công suất, giúp nhận diện các sự kiện tiêu thụ điện năng chậm, yếu hoặc bị che khuất trong môi trường có thiết bị công suất thay đổi mạnh (CVD). Nhờ đó, việc phát hiện các sự kiện trở nên chính xác và ổn định hơn, giảm nhiễu và dao động nền, tạo điều kiện thuận lợi cho các bước xử lý tiếp theo.

\section{Thuật toán Hybrid của Mengqi Lu và Zuyi Li}

Sau khi hoàn tất quá trình tiền xử lý bằng bộ lọc Trung bình (Averaging Filter) và bộ lọc Kalman (Kalman Filter), chúng tôi áp dụng thuật toán Phát hiện Cơ sở (Base Detection Algorithm) – một thành phần của Thuật toán Hybrid được đề xuất bởi Mengqi Lu và Zuyi Li \cite{ref7} – lên dữ liệu đã được làm mượt. Thuật toán này xác định các sự kiện chuyển mạch dựa trên tiêu chí chênh lệch lớn giữa công suất trung bình trước và sau một điểm thời gian quan sát. Chức năng chính của phương pháp này là nâng cao khả năng phát hiện các sự kiện có biên độ nhỏ hoặc dễ bị che khuất trong môi trường có sự biến động công suất mạnh.

Thuật toán Cơ sở hoạt động bằng cách tìm kiếm sự chênh lệch lớn về công suất trung bình đã ước tính ($P_{\text{estimated}}$) giữa hai cửa sổ trượt liền kề.

\textbf{Tính toán Trung bình Động}

Tại mỗi điểm mẫu $i$, thuật toán tính toán công suất trung bình trong hai cửa sổ trượt có kích thước $n$ mẫu:
\begin{align*}
Mean_{i,\text{before}} &= \frac{1}{n} \sum_{j=i-n}^{i-1} P_{\text{estimated}, j} \\
Mean_{i,\text{after}} &= \frac{1}{n} \sum_{j=i+1}^{i+n+1} P_{\text{estimated}, j}
\end{align*}
\textit{Kích thước cửa sổ $n$ được chọn tương ứng với khoảng $10$ giây để tối ưu hóa độ nhạy.}

\textbf{Điều Kiện Phát Hiện Sự Kiện}

Một sự kiện được xác nhận tại thời điểm $i$ nếu độ chênh lệch tuyệt đối giữa hai giá trị trung bình này vượt quá ngưỡng công suất $P_{\text{th}}$:
\begin{equation*}
\text{Sự kiện xảy ra tại } i \text{ nếu } |Mean_{i,\text{after}} - Mean_{i,\text{before}}| > P_{\text{th}}
\end{equation*}

\textbf{Cơ Chế Giới Hạn Thời Gian (Time Limit)}

Để tránh việc phát hiện một quá trình quá độ thành nhiều sự kiện nhỏ, cơ chế Giới hạn Thời gian ($T_{\text{th}}$) được áp dụng. Nếu hai sự kiện được phát hiện liên tiếp trong khoảng thời gian $\Delta t < T_{\text{th}}$ (chọn $T_{\text{th}} = 0.2$ giây theo khuyến nghị của bài báo gốc), chúng sẽ được gộp lại và coi là một sự kiện chuyển trạng thái duy nhất.

Ở trong phạm vi của đồ án này, MLZL là thuật toán Hybrid của Mengqi Lu và Zuyi Li gốc, và MLZLI là cho thuật toán đó nhưng được chỉnh phù hợp áp dụng cho thuật toán đề xuất.

\chapter{Tiền xử lý dữ liệu và trích xuất đặc trưng}

\section{Mục tiêu của giai đoạn tiền xử lý và trích xuất đặc trưng}
Mục tiêu chính của giai đoạn tiền xử lý và trích xuất đặc trưng là xác định tính toán đặc trưng của thiết bị vừa thay đổi trạng thái hoạt động. Quá trình này bao gồm trích xuất hai nhóm đặc trưng quan trọng. Thứ nhất, hệ thống tiến hành trích xuất đặc trưng công suất tiêu thụ tại thời điểm thiết bị xảy ra thay đổi trạng thái, chẳng hạn bật, tắt hoặc chuyển chế độ hoạt động. Thứ hai, hệ thống trích xuất đặc trưng hình ảnh I–V, trong đó mối quan hệ giữa điện thế tức thời~$U$ và cường độ dòng điện tức thời~$I$ được biểu diễn dưới dạng quỹ đạo, với trục hoành là điện áp và trục tung là dòng điện.

Trong hệ thống điện dân dụng, hiệu điện thế~$U$ thường có dạng gần như cố định, trong khi dòng điện~$I$ thay đổi tùy theo đặc tính tải của từng thiết bị. Nhờ đó, mỗi thiết bị tạo ra một đặc trưng I–V riêng biệt, phản ánh hành vi tiêu thụ điện đặc trưng của nó. Các đặc trưng về công suất và đồ thị I–V này được sử dụng làm cơ sở cho quá trình nhận dạng thiết bị, giúp phân biệt các thiết bị điện trong cùng một hệ thống.

\section{Trích xuất đặc trưng công suất}

Khi thuật toán phát hiện sự kiện xác định có sự thay đổi trạng thái xảy ra, nó trả về thời điểm phát hiện sự kiện cùng với độ dài của cửa sổ sự kiện. Dựa trên hai thông tin này, ta xác định hai mốc thời gian quan trọng: Thời điểm trước sự kiện, được lấy sớm hơn thời điểm phát hiện một vài giây, và thời điểm sau sự kiện, được lấy muộn hơn thời điểm kết thúc cửa sổ sự kiện một khoảng thời gian tương tự. Việc dịch chuyển các mốc thời gian như vậy nhằm loại bỏ giai đoạn quá độ, tức khoảng thời gian thiết bị vừa thay đổi trạng thái nhưng công suất chưa ổn định. Nhờ đó, dữ liệu sử dụng để trích xuất đặc trưng phản ánh đúng trạng thái ổn định của thiết bị trước và sau thay đổi.

Sau khi xác định hai khoảng thời gian này, thuật toán tiến hành tính giá trị trung bình của công suất tức thời trong từng khoảng và sau đó lấy hiệu tuyệt đối giữa hai giá trị trung bình. Kết quả thu được chính là đặc trưng công suất chênh lệch tại thời điểm của thiết bị tại thời điểm chuyển trạng thái. Đối với quá trình trích xuất đặc trưng hình ảnh I–V, các mốc thời gian “trước” và “sau” sự kiện cũng được xác định theo cách tương tự nhằm đảm bảo tính nhất quán giữa hai loại đặc trưng.


\section{Trích xuất đặc trưng ảnh}

\subsection{Cơ sở lý thuyết}

Trong hệ thống phân tích phụ tải phi xâm lấn (NILM), toàn bộ phép đo được thực hiện tại một điểm duy nhất trên mạng điện, thường là ngay sau công tơ tổng. Điều này có nghĩa là hệ thống chỉ quan sát được điện áp $U(t)$ và dòng điện tổng $I(t)$ của toàn bộ tải trong nhà, thay vì quan sát từng thiết bị riêng lẻ như các hệ đo đa kênh. Vì vậy, bài toán NILM đặt ra câu hỏi trung tâm: Từ tín hiệu tổng hợp này, làm thế nào để suy ra thiết bị nào vừa thay đổi trạng thái?

Mạng điện dân dụng là mạng mắc song song, do đó điện áp đặt lên tất cả các thiết bị là như nhau. Khi một thiết bị mới được bật, nó được mắc song song với các thiết bị đang hoạt động, dẫn đến tổng dòng điện trong mạch tăng lên. Quan hệ này được mô tả bởi định luật Kirchhoff về dòng điện:

\[
I_{\text{total}} = \sum_{k=1}^{n} I_k.
\]

Giả sử tại thời điểm $t_0$, hệ thống có $n$ thiết bị đang hoạt động, với tổng dòng điện $I_t$. Tại thời điểm $t_1$, khi thiết bị thứ $(n+1)$ được bật, dòng tổng tăng lên thành $I_p$. Vì đặc tính mắc song song, điện áp tổng trước và sau khi bật thiết bị được xem như không đổi:

\begin{equation}
U_{n+1} = U_p = U_t.
\end{equation}

Dòng điện của thiết bị mới có thể được suy ra từ chênh lệch dòng tổng:

\begin{equation}
I_{n+1} = I_p - I_t.
\end{equation}

Hai hình dưới đây minh họa mạng điện trước và sau khi thiết bị mới được cắm vào:

\begin{figure}[H]{Mạng điện trước khi cắm thêm thiết bị $n{+}1$}
    \centering
    \includegraphics[height=0.35\linewidth]{IMAGE/Co_so_ly_thuyet_tien_xu_ly/mangdienn.jpg}
\end{figure}

\begin{figure}[H]{Mạng điện sau khi cắm thêm thiết bị $n{+}1$}
    \centering
    \includegraphics[height=0.35\linewidth]{IMAGE/Co_so_ly_thuyet_tien_xu_ly/mangdienn_1.jpg}
\end{figure}

Từ góc nhìn công suất, ta có:

\[
P(t) = U(t)\,I(t),
\]

và sự kiện bật/tắt thiết bị tạo ra một bước nhảy công suất:

\[
\Delta P = P_{\text{after}} - P_{\text{before}}.
\]

Đây chính là ``chữ ký'' quan trọng giúp các thuật toán phát hiện sự kiện nhận biết thời điểm có thiết bị thay đổi trạng thái.

Khi thiết bị bật, tổng trở tương đương của toàn mạng thay đổi, kéo theo sự thay đổi dòng điện tổng. Mức thay đổi này chịu ảnh hưởng của loại thiết bị, nguyên lý hoạt động (điện trở thuần, thiết bị có cuộn cảm, thiết bị có bộ nguồn switching), và đặc tính chuyển mạch. Nhờ đó, các hệ thống NILM có thể dựa trên dạng thay đổi của $I(t)$ và $P(t)$ để nhận diện thiết bị.

Cơ sở lý thuyết của NILM dựa trên ba nguyên lý quan trọng:

\begin{itemize}
    \item Mạng điện dân dụng mắc song song nên điện áp bằng nhau ở mọi tải.
    \item Dòng điện tổng là tổng dòng của từng thiết bị riêng lẻ.
    \item Sự kiện bật/tắt tạo ra bước nhảy đặc trưng trong tín hiệu dòng hoặc công suất.
\end{itemize}

Những nguyên lý này tạo nền tảng cho toàn bộ các thuật toán phát hiện sự kiện và nhận dạng thiết bị được trình bày ở các phần tiếp theo.

\subsection{Vấn đề thực tế}

Trong hệ thống điện thực tế, dòng điện và điện áp được đo tại từng thời điểm tức thời (instantaneous). Do nguồn điện sử dụng là dòng xoay chiều (AC), giá trị của điện áp và dòng điện luôn biến thiên theo thời gian theo dạng hình sin.
Khi thu thập dữ liệu trong một khoảng thời gian đủ dài trước và sau thời điểm thiết bị thay đổi trạng thái (bật hoặc tắt), ta thu được hai dãy mẫu $U_t$ $I_t$ (trước) và $U_p$ $I_p$ (sau). Tuy nhiên, thời điểm bắt đầu lấy mẫu của hai dãy không đảm bảo trùng pha hoặc cách nhau đúng một số nguyên chu kỳ điện. Điều này có nghĩa là dạng sóng $U_t$ và $U_p$ có thể bị lệch pha so với nhau. Vì lý do đó, ta không thể trực tiếp lấy hiệu tức thời $I_p-I_t$ hay $U_p=U_t$ để trích xuất đặc trưng thay đổi của thiết bị — việc so sánh hai tín hiệu lệch pha sẽ gây sai lệch lớn và không phản ánh đúng sự khác biệt do thiết bị tạo ra.

Ngoài ra, trong điều kiện thực tế, nếu cảm biến hoạt động ở tần số lấy mẫu thấp hoặc không ổn định, các điểm lấy mẫu thậm chí còn có khoảng cách thời gian không đều nhau. Điều này làm cho việc so khớp pha và tái dựng dạng sóng càng trở nên khó khăn, đòi hỏi phải có bước xử lý bổ sung như nội suy, đồng bộ pha hoặc tái đồng bộ tín hiệu trước khi phân tích.

\begin{figure}[H]{Minh họa mảng $Up$ và $Ut$ khác nhau, nên đồ thì chúng cũng khác nhau
}
    \centering
    \includegraphics[width=0.9\linewidth]{IMAGE/Trich xuat dac trung anh/Screenshot 2025-11-25 103337.png}
\end{figure}

\begin{figure}[H]{Hình ảnh minh họa các điểm U được lấy mẫu lấy mẫu không đều}
    \centering
    \includegraphics[width=0.9\linewidth]{IMAGE/Trich xuat dac trung anh/Screenshot 2025-11-25 104309.png}
\end{figure}

Với vấn đề trên, chỉ cần đảm bảo U trước và sau giống nhau, hay cùng pha là có thể thực hiện trừ I và trích xuất đặc trưng

\section{Phương pháp tính}

Quá trình trích xuất đặc trưng ảnh từ tín hiệu điện áp $U$ và dòng điện $I$ gồm bốn bước chính: nội suy dữ liệu, căn chỉnh pha, chia đoạn theo chu kỳ và lấy trung bình, sau đó tính hiệu và biểu diễn dạng sóng đặc trưng. Mỗi bước đều đóng vai trò quan trọng trong việc đảm bảo rằng tín hiệu đầu vào được xử lý một cách nhất quán, giảm nhiễu, và làm nổi bật chính xác phần thay đổi do thiết bị gây ra khi bật/tắt. Dưới đây là mô tả chi tiết từng bước.

\textbf{Bước 1: Nội suy tuyến tính}

Dữ liệu thu thập từ cảm biến thường có tần số lấy mẫu không quá cao hoặc đôi khi lấy mẫu với không cách đều theo thời gian, dẫn đến khó khăn khi cần so sánh hai vùng tín hiệu trước và sau sự kiện. Vì vậy, bước đầu tiên trong quá trình xử lý là thực hiện nội suy tuyến tính cho cả hai tín hiệu~$U$ và~$I$. Việc nội suy này giúp tăng số lượng điểm mẫu, từ đó cải thiện độ phân giải tín hiệu theo thời gian; đồng thời làm mượt dạng sóng, giúp việc quan sát chu kỳ và phát hiện những biến thiên trở nên rõ ràng hơn. Bên cạnh đó, nội suy tuyến tính còn góp phần giảm sai số do hiện tượng trễ hoặc do thiếu mẫu trong quá trình thu thập dữ liệu. Phương pháp này không làm thay đổi đặc tính gốc của tín hiệu mà chỉ tăng mật độ thông tin, giúp các bước căn chỉnh và so khớp ở các giai đoạn tiếp theo được thực hiện chính xác hơn.

\begin{figure}[H]{Ví dụ minh hoạ: nội suy tín hiệu $U$ với số lượng điểm tăng gấp 6 lần số điểm gốc}
    \centering
    \includegraphics[width=0.9\linewidth]{IMAGE/Trich xuat dac trung anh/buoc1.png}
\end{figure}

\textbf{Bước 2: Dịch mảng và căn chỉnh pha}

Mục tiêu của bước này là tìm độ dịch tối ưu để hai mảng điện áp $U_p$ (sau khi thiết thay đổi trạng thái) và $U_t$ (trước khi thiết bị thay đổi trạng thái) trở nên trùng khớp với nhau nhất có thể.

Quy trình thực hiện như sau:
\begin{enumerate}
    \item Dịch mảng $U_t$ theo từng bước nhỏ (theo thời gian hoặc theo chỉ số mẫu).
    \item Tại mỗi vị trí dịch, tính sai số giữa hai mảng bằng công thức:    \begin{equation}
        E = \sum_{i=1}^{n} \left| U_p(i) - U_t(i) \right|
    \end{equation}
    Với n là độ dài mảng
    \item Sai số $E$ biểu thị mức độ khác biệt giữa hai dạng sóng tại vị trí dịch tương ứng.
    \item Lặp lại quá trình dịch và tính toán, ta tìm được vị trí dịch tạo ra sai số nhỏ nhất — đây chính là lúc hai dạng sóng đã gần trùng khớp nhất.
\end{enumerate}

Khi tìm được độ dịch tối ưu:

\begin{itemize}
    \item Hai tín hiệu $U_p$ và $U_t$ được căn chỉnh theo thời gian (đồng bộ pha).
    \item Dòng điện $I_t$ tương ứng vớ $U_t$, nên $I_t$ được dịch theo cùng độ dịch.
    \item Lấy hiệu trực tiếp hai mảng dòng điện: 
    $
       I_{n+1} = I_p - I_t
    $
\end{itemize}
Nhờ đó, phần biến thiên của dòng điện thu được phản ánh đúng thay đổi do thiết bị tạo ra, thay vì sai lệch pha hoặc sai lệch thời gian giữa các mẫu đo.

\begin{figure}[H]{Minh họa Up, Ut trước và sau khi dịch theo Up}
    \centering
    \includegraphics[width=0.9\linewidth]{IMAGE/Trich xuat dac trung anh/buoc2.png}
\end{figure}

\textbf{Bước 3: Chia đoạn và lấy trung bình theo chu kỳ}

Sau khi tín hiệu được căn chỉnh pha, dữ liệu được chia thành nhiều đoạn tương ứng với từng chu kỳ điện, dựa trên các điểm zero-crossing của điện áp hoặc các vị trí cực đại và cực tiểu. Các chu kỳ này sau đó được xếp chồng lên nhau và tính trung bình theo từng vị trí mẫu theo công thức:
\[
\bar{U}(i) = \frac{1}{m} \sum_{k=1}^{m} U_k(i), \qquad 
\bar{I}(i) = \frac{1}{m} \sum_{k=1}^{m} I_k(i).
\]
Quá trình này giúp loại bỏ nhiễu ngẫu nhiên, tăng độ ổn định của dạng sóng và thu được một hình dạng trung bình mang tính đại diện cho thiết bị. Kết quả cuối cùng của bước này là “đặc trưng dạng sóng”, một dạng đặc trưng gần như bất biến theo thời gian và có khả năng phân biệt rõ ràng giữa các thiết bị điện khác nhau.


\begin{figure}[H]{Hình minh họa mô tả quá trình chia đoạn và tính trung bình.}
    \centering
    \includegraphics[width=0.9\linewidth]{IMAGE/Trich xuat dac trung anh/buoc3.png}
\end{figure}

Hoàn thành bước này nghĩa là ta đã trích xuất thành công “đặc trưng dạng sóng” của thiết bị.

\textbf{Bước 4: Lấy hiệu và vẽ đồ thị}

Ở bước 4, ta thực hiện:
\begin{itemize}
    \item Vẽ đồ thị các tín hiệu U và I sau khi đã xử lý.
    \item Tính hiệu trực tiếp: $U_{n+1} = U_p = U_t$, $I_{n+1} = I_p - I_t$
    \item Biểu diễn kết quả bằng đồ thị để quan sát trực quan sự thay đổi tín hiệu tại thời điểm thiết bị hoạt động.
\end{itemize}

Các hình ảnh ở góc dưới bên trái minh họa tín hiệu U và I sau khi hoàn thành bước 3 (trích xuất đặc trưng dạng sóng), trong khi các hình còn lại thể hiện kết quả cuối sau bước 4.

\begin{figure}[H]{Hình ảnh đồ thị từ bước 3 và 4.}
    \centering
    \includegraphics[width=0.8\linewidth]{IMAGE/Trich xuat dac trung anh/buoc4.jpg}
\end{figure}

\textbf{Bước 5: Chuyển đồ thị sang ảnh}

Ở bước cuối, tín hiệu U–I được chuyển đổi từ dạng đồ thị sang dạng ảnh để phù hợp với khối nhận dạng thiết bị. Cụ thể, đồ thị U–I sau khi xử lý sẽ được chuyển thành một ảnh đen trắng có kích thước \(32 \times 32\).

\begin{figure}[H]{Đồ thị U-I được chuyển sang ảnh đặc trưng}
    \centering
    \includegraphics[width=0.8\linewidth]{IMAGE/Trich xuat dac trung anh/dothisanganh.png}
\end{figure}

\section{Kết quả trích xuất đặc trưng đồ thị}

Thực hiện đánh giá bằng cách so sánh đặc trưng ảnh giữa việc dùng thuật toán và không sử dụng thuật toán trong trường hợp chỉ có 1 thiết bị hoạt động duy nhất làm tham chiếu
Chạy thử thuật toán trên một số thiết bị cho kết quả như sau:

Đối với quạt không sử dụng thuật toán
\begin{figure}[H]{Đặc trưng đồ thị của quạt không sử dụng thuật toán và hoạt động một mình}
    \centering
    \includegraphics[height=0.4\linewidth]{IMAGE/Ket_qua_thuat_toan_tao_anh/dactrungquatkothuattoan.png}
\end{figure}

\begin{figure}[H]{Đặc trưng đồ thị của quạt có sử dụng thuật toán tạo ảnh và hoạt động một mình}
    \centering
    \includegraphics[height=0.5\linewidth]{IMAGE/Ket_qua_thuat_toan_tao_anh/dactrungquatthuatoan.png}
\end{figure}

\begin{figure}[H]{Đặc trưng đồ thị của quạt khi dùng cả thuật toán trừ và thuật toán tạo ảnh}
    \centering
    \includegraphics[width=0.8\linewidth]{IMAGE/Ket_qua_thuat_toan_tao_anh/dactrungquatcatru.png}
\end{figure}

Hình trái là đồ thị công suất và khoảng thời gian trước, sau được sử dụng để trích xuất đặc trưng đồ thị. Hình phải là đặc trưng đồ thị được trích xuất ra của quạt.

\begin{figure}[H]{Đặc trưng đồ thị của sạc máy tính không sử dụng thuật toán và hoạt động một mình}
    \centering
    \includegraphics[height=0.6\linewidth]{IMAGE/Ket_qua_thuat_toan_tao_anh/image6.png}
\end{figure}

\begin{figure}[H]{Đặc trưng đồ thị của sạc máy tính có sử dụng thuật toán tạo ảnh và hoạt động một mình}
    \centering
    \includegraphics[height=0.6\linewidth]{IMAGE/Ket_qua_thuat_toan_tao_anh/image7.png}
\end{figure}

\begin{figure}[H]{Đặc trưng đồ thị của sạc máy tính khi dùng cả thuật toán trừ và thuật toán tạo ảnh}
    \centering
    \includegraphics[width=0.9\linewidth]{IMAGE/Ket_qua_thuat_toan_tao_anh/image8.png}
\end{figure}

Hình trái là đồ thị công suất và khoảng thời gian trước, sau được sử dụng để trích xuất đặc trưng đồ thị. Hình phải là đặc trưng đồ thị được trích xuất ra của sạc máy tính.


\section{Nhận xét kết quả}

Từ các kết quả thử nghiệm, có thể nhận thấy rằng khi thiết bị hoạt động độc lập và không có sự chồng chéo giữa các tín hiệu, thuật toán tạo ảnh cho ra kết quả khá tương đồng với ảnh gốc (ảnh không qua xử lý). Các đặc trưng chính như biên dạng dòng điện theo chu kỳ, mức độ biến thiên theo thời gian và hình dáng tổng quát của dạng sóng đều được bảo toàn. Tuy nhiên, do bản chất của quá trình lấy trung bình theo chu kỳ, một số chi tiết biên nhỏ hoặc các đoạn có biến thiên nhanh bị làm mờ nhẹ. Điều này là kết quả tất yếu khi ưu tiên giảm nhiễu và thu được dạng sóng đại diện ổn định.

Khi áp dụng thêm thuật toán trừ (sử dụng cặp tín hiệu trước–sau sự kiện), các ảnh đặc trưng thu được thể hiện hình dạng rất gần với ảnh của trường hợp thiết bị hoạt động đơn lẻ có áp dụng thuật toán tạo ảnh. Điều này chứng tỏ rằng quá trình căn chỉnh pha và trích xuất phần tín hiệu đóng góp riêng của thiết bị đã vận hành đúng và chính xác. Đặc biệt, phần biến thiên của dòng điện $I_{n+1}$ sau khi trừ gần như trùng khớp với đặc trưng thực của thiết bị, dù trong tín hiệu gốc có nhiều thành phần chồng chéo từ các thiết bị khác.

Từ các ví dụ minh họa, có thể khẳng định rằng thuật toán trích xuất đặc trưng dạng sóng hoạt động hiệu quả và giữ được hình dạng đủ rõ ràng để cung cấp cho mô hình học máy. Mặc dù một số thông tin biên độ chi tiết có thể giảm nhẹ khi trung bình, nhưng hình dáng tổng thể và cấu trúc năng lượng theo chu kỳ — vốn là phần quan trọng nhất đối với mô hình phân loại — vẫn được bảo toàn. Điều này cho thấy rằng phương pháp đề xuất phù hợp với bài toán phân loại thiết bị trong hệ thống NILM và có tiềm năng đạt độ chính xác cao khi tích hợp vào mô hình nhận diện.

\chapter{Mô hình học máy}

\section{Giới thiệu}

Trong hệ thống Giám sát tả không xâm lấn (Non-Intrusive Load Monitoring -- NILM), sau khi phát hiện được thời điểm xảy ra một sự kiện bật/tắt, nhiệm vụ quan trọng tiếp theo là xác định thiết bị nào đã tạo ra sự kiện đó. Đối với bài toán nhận dạng thiết bị trong đề tài, mô hình được sử dụng là mạng nơ-ron đa lớp (Multi-Layer Perceptron -- MLP) với tập đặc trưng đầu vào bao gồm ảnh đặc trưng I--V và giá trị công suất trung bình của thiết bị.

MLP là một mô hình học máy có cấu trúc đơn giản nhưng mang lại hiệu quả cao đối với các bài toán phân loại có đầu vào dạng vector cố định. Trong bối cảnh của đề tài, ảnh đặc trưng I--V được chuyển đổi thành dạng ma trận và làm phẳng (flatten) thành vector, sau đó kết hợp với giá trị công suất trung bình $P_{\mathrm{mean}}$. Việc biểu diễn như vậy cho phép mô hình học được sự khác biệt đặc trưng giữa các thiết bị điện dựa trên hành vi tiêu thụ của chúng.

Nhờ đặc điểm gọn nhẹ, tốc độ suy luận nhanh và khả năng triển khai trong thời gian thực, MLP trở thành lựa chọn phù hợp cho bài toán phân loại thiết bị trong hệ thống NILM của đề tài. Phần này sẽ trình bày chi tiết quy trình xây dựng đặc trưng đầu vào, kiến trúc mô hình, phương pháp huấn luyện và đánh giá, qua đó làm rõ vai trò của mô hình MLP trong toàn bộ hệ thống nhận dạng thiết bị.

\section{Đầu vào của mô hình}

Đầu vào của mô hình MLP trong đề tài bao gồm hai loại đặc trưng được xử lý thông qua hai nhánh mạng riêng biệt: (1) ảnh đặc trưng I--V và (2) giá trị công suất $P$ của thiết bị tại thời điểm xảy ra sự kiện bật/tắt. Cả hai đặc trưng này đều được trích xuất từ khối tiền xử lý và trích xuất đặc trưng.

Ảnh I--V được xây dựng dựa trên mối quan hệ giữa dòng điện và điện áp khi thiết bị thay đổi trạng thái. Sau giai đoạn tiền xử lý, dữ liệu được ánh xạ thành ảnh kích thước $32 \times 32$, biểu diễn dạng phân bố đặc trưng của thiết bị. Ảnh được chuẩn hoá và làm phẳng (flatten) thành một vector một chiều trước khi đưa vào nhánh xử lý ảnh của mô hình, tạo ra vector đặc trưng $f_{\mathrm{img}} \in \mathrm{R}^{64}$.

Giá trị công suất $P$ của thiết bị tại thời điểm xảy ra sự kiện được đưa vào nhánh mạng thứ hai. Sau hai lớp fully-connected \cite{ref10}, đặc trưng này được biểu diễn dưới dạng vector $f_{P} \in \mathrm{R}^{16}$.

Cuối cùng, hai vector đặc trưng được ghép lại bằng phép nối:
\[
x = [f_{\mathrm{img}},\; f_{P}] \in \mathrm{R}^{80},
\]
và được đưa vào bộ phân loại cuối cùng để xác định thiết bị tương ứng.


\section{Kiến trúc mô hình MLP}

Mô hình MLP được thiết kế nhằm phân loại thiết bị dựa trên hai loại đặc trưng: ảnh I--V và giá trị công suất $P$. Mô hình có cấu trúc gồm ba phần chính: nhánh xử lý ảnh, nhánh xử lý công suất, và bộ phân loại kết hợp.

\begin{figure}[H]{Kiến trúc mô hình MLP được sử dụng}
    \centering
    \includegraphics[width=0.8\linewidth]{IMAGE/kientrucmlp.jpg}
\end{figure}

\subsection{Nhánh xử lý ảnh I--V}

Nhánh này nhận đầu vào là ảnh I--V đã được làm phẳng (flatten) thành vector một chiều có kích thước $32 \times 32 = 1024$. Nhánh xử lý ảnh gồm bốn lớp fully-connected liên tiếp với hàm kích hoạt ReLU:

\begin{itemize}
    \item Lớp 1: 1024 $\rightarrow$ 512 neuron
    \item Lớp 2: 512 $\rightarrow$ 256 neuron
    \item Lớp 3: 256 $\rightarrow$ 128 neuron
    \item Lớp 4: 128 $\rightarrow$ 64 neuron
\end{itemize}

Sau nhánh này, ảnh I--V được biểu diễn dưới dạng vector đặc trưng $f_{\mathrm{img}} \in \mathbb{R}^{64}$.

\subsection{Nhánh xử lý giá trị công suất}

Nhánh này nhận đầu vào là giá trị công suất $P$ của thiết bị tại thời điểm sự kiện. Nhánh gồm hai lớp fully-connected với ReLU:

\begin{itemize}
    \item Lớp 1: 1 $\rightarrow$ 32 neuron
    \item Lớp 2: 32 $\rightarrow$ 16 neuron
\end{itemize}

Kết quả là vector đặc trưng $f_P \in \mathbb{R}^{16}$.

\subsection{Bộ phân loại kết hợp}

Hai vector đặc trưng $f_{\mathrm{img}}$ và $f_P$ được ghép nối thành vector $x \in \mathbb{R}^{80}$:
\[
x = [f_{\mathrm{img}},\; f_P]
\]

Vector $x$ được đưa vào bộ phân loại gồm hai lớp fully-connected với ReLU:

\begin{itemize}
    \item Lớp 1: 80 $\rightarrow$ 32 neuron
    \item Lớp 2: 32 $\rightarrow$ $C$ neuron (số lớp bằng số thiết bị cần phân loại)
\end{itemize}

Lớp đầu ra trả về vector logits, sau đó có thể áp dụng softmax để tính xác suất dự đoán thiết bị.

\subsection{Hàm mất mát và tối ưu}

Mô hình MLP được huấn luyện để phân loại thiết bị sử dụng Categorical Cross-Entropy làm hàm mất mát:
\[
\mathcal{L} = - \sum_{i=1}^{C} y_i \log(\hat{y}_i),
\]
trong đó $C$ là số lớp (số thiết bị), $y_i$ là nhãn thật và $\hat{y}_i$ là xác suất dự đoán từ mô hình.

Mô hình được tối ưu bằng thuật toán Adam \cite{ref11} với learning rate $1 \times 10^{-3}$, huấn luyện trong 15 epoch với batch size 32, lựa chọn này giúp mạng học hiệu quả đặc trưng từ cả ảnh I--V và giá trị công suất $P$ đồng thời hạn chế tình trạng overfit.

\section{Quy trình huấn luyện}

Quy trình huấn luyện mô hình MLP bao gồm các bước chính sau:

\begin{enumerate}
    \item \textbf{Chuẩn hóa dữ liệu:} Ảnh I--V được chuẩn hóa và làm phẳng thành vector một chiều, trong khi giá trị công suất $P$ cũng được chuẩn hóa, đảm bảo các đặc trưng đầu vào cùng thang đo phù hợp cho mạng MLP.
    
    \item \textbf{Chia tập dữ liệu:} Tập huấn luyện được tạo từ các thiết bị hoạt động độc lập (mỗi thiết bị bật một mình), giúp mạng học chính xác đặc trưng của từng thiết bị. Tập kiểm tra (test) được tạo từ các sự kiện thay đổi trạng thái thiết bị trong môi trường có 0 hoặc nhiều thiết bị khác đang bật, nhằm đánh giá khả năng phân loại trong tình huống thực tế. Dữ liệu cũng được chia thành tập train và validation để theo dõi quá trình huấn luyện và điều chỉnh tham số.
    
    \item \textbf{Huấn luyện theo batch:} Mạng được huấn luyện theo batch với batch size = 32, sử dụng hàm mất mát Categorical Cross-Entropy và optimizer Adam với learning rate $1 \times 10^{-3}$. Quá trình huấn luyện kéo dài 15 epoch, đảm bảo mạng học hiệu quả đặc trưng từ cả ảnh I--V và công suất $P$ đồng thời hạn chế overfit.
    
    \item \textbf{Lưu mô hình:} Sau mỗi epoch, trạng thái mô hình và optimizer được lưu lại (checkpoint) để có thể tiếp tục huấn luyện nếu cần. Sau khi huấn luyện kết thúc, mô hình cuối cùng và bộ mã hóa nhãn (label encoder) được lưu để sử dụng cho dự đoán thiết bị.
\end{enumerate}


\chapter{Kết quả và đánh giá thuật toán}

\section{Quy trình đánh giá}

Trong quá trình đánh giá, thuật toán chỉ nhận một tệp dữ liệu duy nhất làm đầu vào. Tệp này được đọc tuần tự để lấy các giá trị tức thời của điện áp U và dòng điện I. Từ chuỗi tín hiệu đó, thuật toán tiến hành phát hiện những thời điểm xảy ra thay đổi trạng thái của thiết bị, sau đó thực hiện các bước tiền xử lý và trích xuất đặc trưng nhằm tạo đầu vào cho mô hình học máy dùng trong phân loại thiết bị.

Do mỗi tệp dữ liệu chỉ chứa một sự kiện thực duy nhất, nên nếu thuật toán phát hiện nhiều hơn một sự kiện thì các sự kiện thừa được xem như sự kiện giả (false positive). Ngược lại, trong trường hợp thuật toán không phát hiện được sự kiện nào, tệp dữ liệu đó được xem là một sự kiện bị bỏ sót (false negative).

Khi đánh giá mô hình học máy trong bài toán nhận diện sự kiện, chỉ những sự kiện mà thuật toán phát hiện được mới được đưa vào mô hình. Điều này dẫn đến một hạn chế quan trọng: mô hình học máy không thể phản ánh đầy đủ hiệu suất của toàn bộ thuật toán, bởi bản thân nó không “nhìn thấy” các trường hợp sự kiện giả hoặc sự kiện bị bỏ lỡ.

Trong thực tế vận hành, thuật toán có thể tạo ra hai loại lỗi. Loại thứ nhất là sự kiện giả, tức thuật toán báo có sự kiện mặc dù thực tế không có. Loại thứ hai là sự kiện bị bỏ sót, tức sự kiện thực tế đã xảy ra nhưng không được phát hiện. Chẳng hạn, với 100 sự kiện thực, thuật toán chỉ phát hiện được 90 sự kiện, nghĩa là đã bỏ sót 10 sự kiện. Trong số 90 sự kiện đã phát hiện, có tám sự kiện thực chất là giả. Như vậy, mô hình học máy chỉ được đánh giá trên 82 sự kiện hợp lệ mà thuật toán gửi đến, nên không thể phản ánh toàn diện hiệu suất của toàn bộ chuỗi xử lý.

Để khắc phục vấn đề này và cho phép đánh giá thuật toán một cách tổng thể, ta áp dụng một quy ước chung. Những sự kiện giả được gán nhãn thực tế là null, trong khi các sự kiện thật nhưng bị bỏ sót được xem như trường hợp mô hình trả về nhãn null. Nhờ quy ước này, toàn bộ thuật toán — bao gồm cả phần phát hiện sự kiện và phần phân loại bằng mô hình học máy — có thể được đánh giá tương tự như cách đánh giá một mô hình máy học thông thường trên tập dữ liệu đầy đủ.

Để đánh giá hiệu quả của thuật toán NILM, các chỉ số phổ biến trong lĩnh vực phân loại và phát hiện sự kiện được sử dụng, bao gồm:

\textbf{Accuracy (Độ chính xác):}
\[
\text{Accuracy} = \frac{TP + TN}{TP + TN + FP + FN}
\]
Chỉ số này phản ánh tỷ lệ dự đoán đúng trên tổng số mẫu.

\textbf{Precision (Độ chính xác dương):}
\[
\text{Precision} = \frac{TP}{TP + FP}
\]
Cho biết mức độ tin cậy khi thuật toán dự đoán rằng có sự kiện xảy ra.

\textbf{Recall (Độ bao phủ):}
\[
\text{Recall} = \frac{TP}{TP + FN}
\]
Thể hiện khả năng phát hiện đúng tất cả các sự kiện thực sự xảy ra.

\textbf{F1-Score:}
\[
F1 = 2 \cdot \frac{\text{Precision} \cdot \text{Recall}}{\text{Precision} + \text{Recall}}
\]
Là trung bình điều hòa giữa Precision và Recall, được dùng khi cần cân bằng giữa hai yếu tố.

\textbf{Confusion Matrix (Ma trận nhầm lẫn):}

Ma trận thể hiện mức độ phân loại đúng sai của mô hình:
\[
\begin{array}{c|cc}
 & \text{Dự đoán: Có} & \text{Dự đoán: Không} \\
\hline
\text{Thực tế: Có} & TP & FN \\
\text{Thực tế: Không} & FP & TN \\
\end{array}
\]

Trong đó:  
- $TP$: Dự đoán đúng sự kiện xảy ra  
- $FP$: Dự đoán nhầm sự kiện  
- $FN$: Bỏ sót sự kiện  
- $TN$: Dự đoán đúng không có sự kiện  

Các chỉ số trên cung cấp cái nhìn toàn diện về hiệu suất thuật toán, bao gồm khả năng phát hiện sự kiện, mức độ nhầm lẫn và mức độ ổn định của mô hình.

\section{Kết quả thực nghiệm}

Tổng số sự kiện cần nhận diện: 116 sự kiện

\subsection{Kết quả thuật toán sử dụng thuật toán phát hiện sự kiện WAMMA}

\begin{table}[H]{Kết quả thuật toán phát hiện sự kiện WAMMA}
    \centering
    \begin{tabular}{|l|c|}
        \hline
        \textbf{Chỉ số đánh giá} & \textbf{Giá trị} \\ \hline
        Số sự kiện được phát hiện chính xác & 65/116 (56\%) \\ \hline
        Số sự kiện bị bỏ lỡ & 51/116 (44\%) \\ \hline
        Số lượng sự kiện giả phát sinh & 2 \\ \hline
    \end{tabular}
\end{table}

\begin{table}[H]{Kết quả mô hình và toàn bộ thuật toán sử dụng thuật toán WAMMA}
    \centering
    \begin{tabular}{|l|c|c|}
        \hline
        \textbf{Chỉ số} & \textbf{Chỉ mô hình học máy} & \textbf{Toàn bộ thuật toán} \\ \hline
        F1 Score & 95\% & 62\% \\ \hline
        Accuracy & 95.24\% & 50.85\% \\ \hline
    \end{tabular}
\end{table}

\begin{figure}[H]{Kết quả phân loại sử dụng thuật toán WAMMA chỉ xét đến mô hình}
    \centering
    \includegraphics[width=0.8\linewidth]{IMAGE/ket_qua_he_thong/image9.png}
\end{figure}

\begin{figure}[H]{Kết quả toàn bộ thuật toán sử dụng thuật toán sự kiện WAMMA}
    \centering
    \includegraphics[width=0.8\linewidth]{IMAGE/ket_qua_he_thong/image10.png}
\end{figure}

\subsection{Kết quả thuật toán sử dụng thuật toán phát hiện sự kiện Hybrid của Mengqi Lu và Zuyi Li}
\begin{table}[H]{Kết quả thuật toán phát hiện sự kiện Hybrid của Mengqi Lu và Zuyi Li}
    \centering
    \begin{tabular}{|l|c|}
        \hline
        \textbf{Chỉ số đánh giá} & \textbf{Giá trị} \\ \hline
        Số sự kiện được phát hiện chính xác & 81/116 (69.83\%) \\ \hline
        Số sự kiện bị bỏ lỡ & 36/116 (30.17\%) \\ \hline
        Số lượng sự kiện giả phát sinh & 9 \\ \hline
    \end{tabular}
\end{table}

\begin{table}[H]{Kết quả mô hình và toàn bộ thuật toán sử dụng thuật toán phát hiện sự kiện Hybrid của Mengqi Lu và Zuyi Li}
    \centering
    \begin{tabular}{|l|c|c|}
        \hline
        \textbf{Chỉ số} & \textbf{Chỉ mô hình học máy} & \textbf{Toàn bộ thuật toán} \\ \hline
        F1 Score & 99\% & 73\% \\ \hline
        Accuracy & 98.75\% & 67.52\% \\ \hline
    \end{tabular}
\end{table}

\begin{figure}[H]{Kết quả phân loại sử dụng thuật toán Hybrid của Mengqi Lu và Zuyi Li chỉ xét đến mô hình}
    \centering
    \includegraphics[width=0.8\linewidth]{IMAGE/ket_qua_he_thong/image11.png}
\end{figure}

\begin{figure}[H]{Kết quả toàn bộ thuật toán sử dụng thuật toán sự kiện Hybrid của Mengqi Lu và Zuyi Li}
    \centering
    \includegraphics[width=0.8\linewidth]{IMAGE/ket_qua_he_thong/image12.png}
\end{figure}

\subsection{Kết quả thuật toán sử dụng thuật toán phát hiện sự kiện đề xuất}

\begin{table}[H]{Kết quả thuật toán phát hiện sự kiện đề xuất}
    \centering
    \begin{tabular}{|l|c|}
        \hline
        \textbf{Chỉ số đánh giá} & \textbf{Giá trị} \\ \hline
        Số sự kiện được phát hiện chính xác & 116/116 (100\%) \\ \hline
        Số sự kiện bị bỏ lỡ & 0/116 (0\%) \\ \hline
        Số lượng sự kiện giả phát sinh & 9 \\ \hline
    \end{tabular}
\end{table}

\begin{table}[H]{Kết quả mô hình và toàn bộ thuật toán sử dụng thuật toán phát hiện sự kiện đề xuất}
    \centering
    \begin{tabular}{|l|c|c|}
        \hline
        \textbf{Chỉ số} & \textbf{Chỉ mô hình học máy} & \textbf{Toàn bộ thuật toán} \\ \hline
        F1 Score & 95\% & 84\% \\ \hline
        Accuracy & 95.69\% & 88.8\% \\ \hline
    \end{tabular}
\end{table}

\begin{figure}[H]{Kết quả phân loại sử dụng thuật toán đề xuất chỉ xét đến mô hình}
    \centering
    \includegraphics[width=0.8\linewidth]{IMAGE/ket_qua_he_thong/image13.png}
\end{figure}


\begin{figure}[H]{Kết quả toàn bộ thuật toán sử dụng thuật toán sự kiện đề xuất}
    \centering
    \includegraphics[width=0.8\linewidth]{IMAGE/ket_qua_he_thong/image14.png}
\end{figure}

\section{Nhận xét kết quả}
Kết quả thực nghiệm cho thấy hiệu suất tổng thể của thuật toán NILM hướng sự kiện phụ thuộc rất lớn vào chất lượng của thuật toán phát hiện sự kiện. Mặc dù mô hình học máy dùng để phân loại thiết bị hoạt động ổn định ở cả ba thuật toán, mức độ khác biệt về độ chính xác end-to-end chủ yếu xuất phát từ số lượng sự kiện bị bỏ sót và số lượng sự kiện giả được sinh ra trong giai đoạn phát hiện.

Trước hết, thuật toán WAMMA thể hiện nhiều hạn chế khi chỉ phát hiện được 56\% tổng số sự kiện. Mặc dù mô hình phân loại trên các sự kiện phát hiện được vẫn đạt F1-Score xấp xỉ 95\%, số lượng sự kiện bị bỏ sót lớn khiến thuật toán mất khả năng theo dõi chính xác trạng thái thiết bị theo thời gian. Kết quả này phản ánh đặc trưng của các thuật toán phát hiện dựa trên biên độ dòng điện, vốn dễ bị suy giảm độ nhạy trong điều kiện nhiễu cao hoặc khi thay đổi công suất nhỏ.

Thuật toán Hybrid của Mengqi Lu và Zuyi Li cho thấy sự cải thiện rõ rệt khi nâng tỷ lệ phát hiện lên gần 70\%. Tuy nhiên, việc vẫn bỏ lỡ hơn 30\% sự kiện cùng với 9 sự kiện giả khiến hiệu suất end-to-end chỉ đạt 67.52\%. Điều này cho thấy mặc dù Hybrid ổn định hơn WAMMA, nhưng vẫn chưa đạt độ tin cậy cần thiết để triển khai trong bối cảnh thực tế có nhiều thiết bị hoạt động đồng thời.

Thuật toán đề xuất trong đề tài đạt kết quả nổi bật nhất khi phát hiện đầy đủ 116/116 sự kiện (100\%), đảm bảo không bỏ sót bất cứ sự kiện nào -- yếu tố then chốt trong các thuật toán NILM hướng sự kiện. Nhờ đảm bảo tính đầy đủ của dữ liệu đầu vào, mô hình phân loại duy trì hiệu suất ổn định và thuật toán không bị đứt gãy dòng sự kiện. Tuy nhiên, thuật toán vẫn sinh ra 9 sự kiện giả, chủ yếu do nhiễu tín hiệu trong quá trình thiết bị CVD thay đổi công suất hoặc do ngưỡng phát hiện chưa được tối ưu cho các thiết bị có đặc trưng đặc thù.

Mặc dù tồn tại các sự kiện giả, việc sử dụng thuật toán đề xuất đạt hiệu suất tổng thể cao nhất với Accuracy 88.8\%, Recall 89\% và F1-Score 84\%. Điều này chứng minh hiệu quả của cách tiếp cận ưu tiên không bỏ sót sự kiện (high-recall strategy), một yêu cầu quan trọng đối với các ứng dụng giám sát thiết bị theo thời gian thực, phân tích phụ tải, hay tối ưu hóa năng lượng.

Nhìn chung, các kết quả thu được khẳng định rằng thuật toán phát hiện sự kiện đóng vai trò quyết định trong thuật toán NILM hướng sự kiện. Mô hình học máy chỉ có thể đạt hiệu suất cao khi giai đoạn phát hiện đảm bảo tính đầy đủ và chính xác của dữ liệu đầu vào. Đồng thời, kết quả cũng cho thấy bộ dữ liệu thực nghiệm của đề tài -- phản ánh điều kiện điện áp và môi trường nhiễu thực tế -- là một thách thức cao hơn so với nhiều bộ dữ liệu công bố trước đây, và vì vậy giúp kiểm chứng mức độ ứng dụng thực tế của thuật toán.

Tổng hợp lại, thuật toán NILM hướng sự kiện được đề xuất đã chứng minh:
\begin{itemize}
    \item Tính khả thi trong môi trường thực nghiệm,
    \item Khả năng mở rộng cho các thuật toán giám sát năng lượng thông minh,
    \item Hiệu quả của việc kết hợp xử lý tín hiệu với mô hình học máy đơn giản nhưng ổn định.
\end{itemize}

Tuy nhiên, để hướng tới triển khai thực tế, các vấn đề như giảm số lượng sự kiện giả, tối ưu hóa ngưỡng phát hiện theo từng loại thiết bị, và mở rộng đánh giá sang dữ liệu dài hạn hoặc dữ liệu thời gian thực cần tiếp tục được nghiên cứu trong các hướng phát triển tiếp theo.

\chapter{Hướng nghiên cứu tương lai}

Trong phạm vi đồ án, hệ thống được triển khai theo hướng xử lý ngoại tuyến, cho phép sử dụng toàn bộ tín hiệu đã thu thập để phát hiện sự kiện và trích xuất đặc trưng. Tuy nhiên, khi ứng dụng ngoài thực tế, một hệ thống NILM phải hoạt động liên tục theo thời gian thực, đòi hỏi nhiều thay đổi quan trọng trong cấu trúc xử lý, cơ chế tổ chức dữ liệu và phương pháp nhận dạng thiết bị. Vì vậy, việc nghiên cứu và hoàn thiện các thành phần của hệ thống trong điều kiện thời gian thực là hướng phát triển quan trọng trong giai đoạn tiếp theo.

\section{Triển khai hệ thống NILM theo thời gian thực}

Trong môi trường thực tế, hệ thống cần một bộ đệm trượt luôn duy trì dữ liệu dòng điện \(I\) và điện áp \(U\) gần nhất. Cơ chế này cho phép hệ thống giám sát liên tục và phát hiện ngay các dấu hiệu bất thường trong tín hiệu, chẳng hạn như thay đổi công suất, nhiễu chuyển tiếp hoặc méo dạng sóng. Khi sự kiện bật hoặc tắt thiết bị xảy ra, hệ thống phải lập tức trích xuất một cửa sổ tín hiệu bao gồm cả vùng trước và sau thời điểm thay đổi, đồng thời chuyển chúng sang các khối phân tích đặc trưng và nhận dạng thiết bị.

Quá trình vận hành liên tục này đòi hỏi phần cứng phải có khả năng lấy mẫu với tốc độ cao, xử lý nhanh và sở hữu dung lượng bộ nhớ đủ lớn để duy trì bộ đệm mà không làm mất dữ liệu. Bên cạnh đó, việc đồng bộ thời gian giữa các sự kiện là yêu cầu quan trọng nhằm tránh tình trạng gộp nhầm các sự kiện xảy ra gần nhau hoặc bỏ sót những biến động nhỏ. Những yếu tố này cho thấy việc triển khai NILM trong môi trường thực phức tạp hơn đáng kể so với mô phỏng ở đồ án, song mô hình hiện tại vẫn đóng vai trò nền tảng quan trọng cho việc phát triển hệ thống thời gian thực trong tương lai.

\section{Hạn chế của mô hình hiện tại khi số lượng thiết bị lớn}

Công suất tiêu thụ là một trong những đặc trưng quan trọng nhất trong nhận dạng thiết bị NILM. Khi số lượng thiết bị ít và sự khác biệt công suất lớn, mô hình thường nhận dạng chính xác vì biên phân tách rõ ràng. Tuy vậy, khi số lượng thiết bị tăng lên, sự chồng lấn giữa các đặc trưng trở nên nghiêm trọng hơn. Trong nhiều trường hợp, hệ thống có thể nhầm lẫn giữa các thiết bị hoàn toàn khác nhau về công suất, chẳng hạn như sạc laptop bị nhận nhầm thành máy ép trái cây hoặc máy sấy có thể bị nhận dạng thành quạt.

Bên cạnh đó, mô hình học máy phải gánh toàn bộ tập thiết bị cùng lúc, khiến kích thước mô hình ngày càng lớn, khó huấn luyện và khó mở rộng khi bổ sung thiết bị mới. Một vấn đề khác là khi hệ thống gặp một thiết bị chưa từng xuất hiện trong quá trình huấn luyện, mô hình có xu hướng gán nhầm thiết bị đó vào một lớp bất kỳ có công suất khác biệt lớn, gây ra sai số nghiêm trọng và làm giảm độ tin cậy của toàn hệ thống.

\section{Hướng tiếp cận mới: Phân nhóm thiết bị theo công suất}

Một hướng nghiên cứu tiềm năng nhằm giải quyết các hạn chế trên là chia bài toán nhận dạng thành nhiều mô hình nhỏ, mỗi mô hình chỉ học phân biệt các thiết bị trong một khoảng công suất nhất định. Thay vì xây dựng một mô hình duy nhất cho toàn bộ thiết bị, hệ thống có thể tổ chức thiết bị thành các nhóm công suất, ví dụ như nhóm công suất thấp, trung bình hoặc cao. Mỗi nhóm được huấn luyện một mô hình riêng, chỉ cần phân biệt các thiết bị nằm trong cùng khoảng giá trị, nơi khoảng cách công suất thường nhỏ và đặc trưng tương đồng hơn.

Khi một sự kiện xảy ra, hệ thống trước hết ước lượng công suất tức thời của thiết bị gây ra sự kiện. Giá trị công suất này được sử dụng để chọn mô hình phù hợp. Nhờ đó, mô hình chỉ cần so sánh giữa các thiết bị trong cùng nhóm công suất, giảm đáng kể khả năng nhầm lẫn giữa các thiết bị thuộc các khoảng công suất rất khác nhau. Nếu công suất đo được nằm ngoài phạm vi của các nhóm đã định nghĩa, hệ thống có thể trả về trạng thái ``thiết bị chưa xác định'', giúp tránh việc gán sai và tăng độ tin cậy của toàn bộ quy trình.

Cách tiếp cận này giúp mô hình nhận dạng nhẹ hơn, dễ huấn luyện hơn và dễ mở rộng trong tương lai. Khi một thiết bị mới xuất hiện, người phát triển chỉ cần xác định công suất của thiết bị đó và đưa nó vào nhóm công suất tương ứng mà không ảnh hưởng đến các mô hình còn lại. Những thiết bị có công suất nằm ở ranh giới giữa hai nhóm cũng có thể xuất hiện ở nhiều nhóm để tăng tính ổn định trong dự đoán.

% ============================
% TÀI LIỆU THAM KHẢO
% ============================
\bibliographystyle{IEEEtran}
\begin{thebibliography}{11}
\begin{bibsection}{}

\bibitem{ref1}
G.~W. Hart, “Nonintrusive appliance load monitoring,” \emph{Proceedings of the IEEE},
vol.~80, no.~12, pp. 1870--1891, 1992, doi: 10.1109/5.192069.

\bibitem{ref2}
M.~Kaselimi, E.~Protopapadakis, A.~Voulodimos, N.~Doulamis, and A.~Doulamis,
“Towards trustworthy energy disaggregation: A review of challenges, methods, and perspectives for non-intrusive load monitoring,”
\emph{Sensors}, vol.~22, no.~15, p. 5872, Aug. 2022, doi: 10.3390/s22155872.

\bibitem{ref3}
A.~Zoha, A.~Gluhak, M.~Imran, and S.~Rajasegarar,
“Non-intrusive load monitoring approaches for disaggregated energy sensing: A survey,”
\emph{Sensors}, vol.~12, no.~12, pp. 16838--16866, Dec. 2012, doi: 10.3390/s121216838.

\bibitem{ref4}
Shanghai Belling Co., Ltd., \emph{BL0940 Calibration-free Metering IC Datasheet}, 2021.
[Online]. 

\bibitem{ref5}
L.~Yan, W.~Tian, H.~Wang, X.~Hao, and Z.~Li,
“Robust event detection for residential load disaggregation,”
\emph{Applied Energy}, vol.~331, p. 120339, Feb. 2023, doi: 10.1016/j.apenergy.2022.120339.

\bibitem{ref6}
R.~E. Kalman, “A new approach to linear filtering and prediction problems,”
\emph{Journal of Basic Engineering}, vol.~82, no.~1, pp. 35--45, Mar. 1960, doi: 10.1115/1.3662552.

\bibitem{ref7}
M.~Lu and Z.~Li,
“A hybrid event detection approach for non-intrusive load monitoring,”
\emph{IEEE Transactions on Smart Grid}, vol.~11, no.~1, pp. 528--540, Jan. 2020, doi: 10.1109/TSG.2019.2924862.

\bibitem{ref8}
Y.~Liu, X.~Wang, and W.~You,
“Non-intrusive load monitoring by voltage–current trajectory enabled transfer learning,”
\emph{IEEE Transactions on Smart Grid}, vol.~10, no.~5, pp. 5609--5619, Sep. 2019,
doi: 10.1109/TSG.2018.2888581.

\bibitem{ref9}
A.~Paszke \emph{et al.},
“PyTorch: An imperative style, high-performance deep learning library,”
in \emph{Proc. 33rd Int. Conf. Neural Information Processing Systems}, 2019,
pp. 8026--8037. [Online].

\bibitem{ref10}
V.~Nair and G.~E. Hinton,
“Rectified linear units improve restricted Boltzmann machines,”
in \emph{Proc. 27th Int. Conf. Machine Learning (ICML’10)},
Madison, WI, USA: Omnipress, 2010, pp. 807--814.

\bibitem{ref11}
D.~P. Kingma and J.~Ba,
“Adam: A method for stochastic optimization,”
\emph{arXiv preprint arXiv:1412.6980}, 2014, doi: 10.48550/arXiv.1412.6980.

\end{bibsection}
\end{thebibliography}


\end{document} 